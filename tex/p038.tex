\upaper{38}{ДУХИ-ПОМОЩНИКИ ЛОКАЛЬНОЙ ВСЕЛЕННОЙ}
\uminitoc{ПРОИСХОЖДЕНИЕ СЕРАФИМОВ}
\uminitoc{ПРИРОДА АНГЕЛОВ}
\uminitoc{НЕРАСКРЫТЫЕ АНГЕЛЫ}
\uminitoc{МИРЫ СЕРАФИМОВ}
\uminitoc{ОБУЧЕНИЕ СЕРАФИМОВ}
\uminitoc{СЕРАФИЧЕСКАЯ ОРГАНИЗАЦИЯ}
\uminitoc{ХЕРУВИМЫ И САНОВИМЫ}
\uminitoc{ЭВОЛЮЦИЯ ХЕРУВИМОВ И САНОВИМОВ}
\uminitoc{ПРОМЕЖУТОЧНЫЕ СОЗДАНИЯ}
\author{Мелхиседек}
\vs p038 0:1 Существуют три различные категории личностей Бесконечного Духа. Пылкий апостол\fnst{А именно Пётр, см.\,\bibref[139:3.4]{p139 3:4}, \bibref[158:7.3]{p158 7:3}, \bibref[179:3.5]{p179 3:5}, \bibref[192:1.3]{p192 1:3}. Цитата из 1\,Петра~3:22.} понимал это, когда писал об Иисусе, <<который взошёл на небо и пребывает по правую руку от Бога, и которому покорились ангелы, власти и силы>>. Ангелы~--- это духи\hyp{}помощники времени; власти~--- воинства посланников пространства; силы~--- высшие личности Бесконечного Духа.
\vs p038 0:2 \pc Как супернафимы в центральной вселенной и секонафимы в сверхвселенной, так и серафимы со связанными с ними херувимами и сановимами составляют ангельский корпус локальной вселенной.
\vs p038 0:3 Все серафимы довольно единообразны по строению. От вселенной к вселенной, по всем семи сверхвселенным они демонстрируют минимум вариаций; это наиболее стандартные из всех духовных типов личностных существ. Их различные категории составляют корпус искусных и рядовых служителей локальных творений.
\usection{ПРОИСХОЖДЕНИЕ СЕРАФИМОВ}
\vs p038 1:1 Серафимы создаются Вселенским Материнским Духом и планируются структурной единицей~--- 41\,472\fnst{Одна единица = 12 батальонам, см.\,\bibref[38:6.1]{p038 6:1}.} за один раз~--- с момента создания <<ангелов\hyp{}образцов>> и определённых ангельских архетипов в ранние времена Небадона. Сын Создатель и вселенский представитель Бесконечного Духа сотрудничают в создании большого числа Сынов и других вселенских личностей. По завершении этого объединённого усилия Сын приступает к созданию Материальных Сынов, первых созданий, имеющих пол, в то время как Вселенский Материнский Дух одновременно предпринимает своё первоначальное одиночное усилие в воспроизводстве духа. Так начинается создание серафических воинств локальной вселенной.
\vs p038 1:2 Эти ангельские категории проектируются во время планирования эволюции смертных волевых созданий. Создание серафимов восходит к обретению Вселенским Материнским Духом относительной личности, не как равной Сыну Властелину, что соответствует более позднему времени, но как помощницы Сына Создателя на раннем этапе. До этого события серафимы для несения службы в Небадоне были временно предоставлены соседней вселенной.
\vs p038 1:3 Периодическое создание серафимов продолжается; вселенная Небадон всё ещё находится в процессе становления. Вселенский Материнский Дух никогда не прекращает созидательной деятельности в растущей и совершенствующейся вселенной.
\usection{ПРИРОДА АНГЕЛОВ}
\vs p038 2:1 У ангелов нет материальных тел, но они являются отдельными и дискретными существами; они имеют природу и происхождение духа. Хотя и невидимые для смертных, они воспринимают вас такими, какие вы есть во плоти, без помощи преобразователей или переводчиков; они интеллектуально понимают образ жизни смертных и разделяют все нечувственные эмоции и настроения человека. Они ценят и очень радуются вашим успехам в музыке, искусстве и настоящем юморе. Они полностью осознают вашу нравственную борьбу и духовные трудности. Они любят людей, и твои усилия понять и полюбить их могут быть только во благо.
\vs p038 2:2 \pc Хотя серафимы~--- очень нежные и чуткие существа, они не являются созданиями, которым свойственны сексуальные эмоции. Они являются во многом такими же, какими станете вы в обительских мирах, где вы <<не будете ни жениться, ни выходить замуж, но будете как ангелы небесные>>. Ибо все, кто <<удостоится достичь обительских миров, не женятся и не выходят замуж; и не умирают они уже больше, ибо они равны ангелам>>. Тем не менее когда мы имеем дело с разнополыми созданиями, у нас принято говорить о существах, имеющих более прямое происхождение от Отца и Сына, как о сынах Бога, а детей Духа называть дочерьми Бога. Поэтому на планетах разнополых существ ангелы обычно обозначаются местоимениями женского рода.
\vs p038 2:3 Серафимы созданы таким образом, чтобы функционировать как на духовном, так и на буквальном уровнях. Вряд ли найдутся фазы моронтийной или духовной деятельности, которые недоступны для их служения. Хотя по личному статусу ангелы не так уж далеки от людей, в определённых функциональных проявлениях серафимы далеко их превосходят. Они обладают многими способностями, намного превышающими человеческое понимание. Например: тебе было сказано, что <<у тебя и волосы на голове все сочтены>>, и это правда, но серафим не тратит своё время на то, чтобы считать и постоянно уточнять их число. Ангелы обладают врождёнными и автоматическими (то есть автоматическими в вашем восприятии) способностями знать такие вещи; вы действительно считали бы серафима математическим гением. Поэтому многочисленные обязанности, которые были бы колоссальными задачами для смертных, выполняются серафимами с чрезвычайной лёгкостью.
\vs p038 2:4 \pc Ангелы превосходят вас в духовном статусе, но они не являются вашими судьями или обвинителями. Каковы бы ни были ваши недостатки, <<ангелы, хотя и превосходят вас в силе и могуществе, не выдвигают против вас никаких обвинений>>. Ангелы не судят человечество, и отдельным смертным не следовало бы осуждать своих собратьев.
\vs p038 2:5 \pc Ты сделаешь правильно, если полюбишь их, но ты не должен преклоняться перед ними; ангелы~--- не объект для поклонения. Великий серафим Лоялация, когда ваш пророк\fnst{Это был Иоанн Зеведеев, а цитата взята из Откровения~22:8--9.} <<пал, чтобы поклониться, к ногам ангела>>, сказал: <<Смотри, не делай этого; ибо я такой же слуга, как ты и братья твои, и всем нам предписано поклоняться Богу>>.
\vs p038 2:6 В отношении природы и личностного дара серафимы лишь немного опережают смертные расы по шкале существования созданий. Действительно, освобождаясь от плоти, ты становишься очень похожим на них. На обительских мирах ты начнёшь ценить серафимов, на сферах созвездия~-- радоваться им, а на Спасограде они будут делить с тобой свои места отдыха и поклонения. На протяжении всего моронтийного и последующего духовного восхождения твоё братство с серафимами будет идеальным, а дружба~--- возвышенной.
\usection{НЕРАСКРЫТЫЕ АНГЕЛЫ}
\vs p038 3:1 По всем областям локальной вселенной функционируют многочисленные категории духовных существ, не раскрытых смертным, поскольку они никоим образом не связаны с эволюционным планом Райского восхождения. В этом документе слово <<ангел>> преднамеренно ограничено обозначением тех серафических и связанных с ними потомков Вселенского Материнского Духа, которые главным образом заняты выполнением планов выживания смертных. В локальной вселенной служат шесть других категорий родственных существ, нераскрытых ангелов, которые не связанны каким\hyp{}либо образом с вселенской деятельностью, касающейся восхождения к Раю эволюционных смертных. Эти шесть групп ангельских партнёров никогда не называют серафимами; никогда не говорят о них и как о духах\hyp{}помощниках. Эти личности целиком заняты административными и другими делами Небадона, занятиями, не имеющими отношения к постепенному пути духовного восхождения и достижения совершенства человека.
\usection{МИРЫ СЕРАФИМОВ}
\vs p038 4:1 Девятая группа семи первичных сфер контура Спасограда~--- это миры серафимов. У каждого из этих миров есть шесть подчинённых спутников, на которых расположены специальные школы, посвящённые всем фазам серафического обучения. Хотя серафимы имеют доступ ко всем 49 мирам, составляющим данную группу сфер Спасограда, они занимают исключительно первое скопление из семи. Остальные шесть скоплений заняты шестью категориями ангельских партнёров, не раскрытых на Урантии; каждая из этих групп имеет свой центр на одном из этих шести первичных миров и осуществляет специализированную деятельность на шести подчинённых спутниках. Каждая ангельская категория имеет свободный доступ ко всем мирам этих семи разнообразных групп.
\vs p038 4:2 Эти центральные миры входят в число великолепнейших сфер Небадона; серафические владения отличаются как красотой, так и простором. Здесь каждый серафим имеет настоящий дом, а <<дом>> означает жилище двух серафимов; они живут парами.
\vs p038 4:3 \pc Хотя они не являются ни мужчинами, ни женщинами, как Материальные Сыны и смертные расы, серафимы бывают отрицательными и положительными. В большинстве назначений для выполнения задачи требуются два ангела. Когда они не подключены к контуру, они могут работать в одиночку; не требуют они дополняющего существа и в стационарном состоянии. Обычно они сохраняют свои первоначальные дополняющие существа, но не всегда. Такие союзы прежде всего необходимы для выполнения функции; им не свойственны сексуальные чувства, хотя они и являются чрезвычайно личностными и истинно нежными.
\vs p038 4:4 Помимо специально отведённых домов у серафимов также есть центры группы, роты, батальона и единицы. Они собираются для воссоединения каждое тысячелетие, и все присутствуют в соответствии с временем своего создания. Если серафим несёт обязанности, запрещающие отсутствие на службе, она чередует присутствие со своим дополнением, будучи заменена серафимом другой даты рождения. Таким образом, каждый серафический партнёр присутствует, по крайней мере, на каждом втором воссоединении.
\usection{ОБУЧЕНИЕ СЕРАФИМОВ}
\vs p038 5:1 Серафимы проводят своё первое тысячелетие в качестве свободных от задания наблюдателей на Спасограде и связанных с ним школах\hyp{}мирах. Второе тысячелетие проходит на серафических мирах контура Спасограда. Их центральная образовательная школа сейчас управляется первыми 100\,000 серафимов Небадона, во главе которых стоит изначальный, или первородный, ангел этой локальной вселенной. Первая созданная группа серафимов Небадона была обучена корпусом из 1\,000 серафимов Авалона; впоследствии наши ангелы обучались своими собственными старшими собратьями. Мелхиседеки также играют большую роль в образовании и подготовке всех ангелов локальной вселенной~--- серафимов, херувимов и сановимов.
\vs p038 5:2 По окончании этого периода обучения на серафических мирах Спасограда серафимы мобилизуются обычными группами и подразделениями ангельской организации и прикрепляются к какому\hyp{}либо из созвездий. Они ещё не назначаются духами\hyp{}помощниками, хотя уже уверенно вступили в фазы ангельского обучения, предшествующие этому назначению.
\vs p038 5:3 Серафимы посвящаются в духи\hyp{}помощники через служение в качестве наблюдателей на низшем из эволюционных миров. После этого опыта они возвращаются на миры, относящиеся к столице созвездия, в котором они получили назначение, чтобы начать своё углублённое обучение и более определённо подготовиться к служению в какой\hyp{}либо конкретной локальной системе. Вслед за этим общим образованием они продвигаются на службу в одну из локальных систем. На архитектурных мирах, связанных со столицей какой\hyp{}либо системы Небадона, наши серафимы завершают своё обучение и назначаются духами\hyp{}помощниками времени.
\vs p038 5:4 Будучи назначенными духами\hyp{}помощниками, серафимы, выполняя задание, могут перемещаться по всему Небадону, даже Орвонтону. Их работа во вселенной не имеет границ и ограничений; они тесно связаны с материальными созданиями миров и вечно служат низшим категориям духовных личностей, осуществляя контакт между этими существами мира духа и смертными материальных сфер.
\usection{СЕРАФИЧЕСКАЯ ОРГАНИЗАЦИЯ}
\vs p038 6:1 После второго тысячелетия временного пребывания на серафическом центре серафимы организуются в группы, подчинённые руководителям, по 12 (12 пар, 24 серафима), и 12 таких групп составляют роту (144 пары, 288 серафимов) под командованием лидера. Двенадцать рот, подчинённые командиру, составляют батальон (1\,728 пар, или 3\,456 серафимов), а 12 батальонов, подчинённые директору, равны серафической единице (20\,736 пар, или 41\,472 индивидуума), в то время как 12 единиц, подчинённые управляющему, составляют легион, насчитывающий 248\,832 пары, или 497\,664 индивидуума. Иисус имел в виду именно такую группу ангелов в ту ночь в Гефсиманском саду, когда сказал: <<Я могу и сейчас попросить Отца моего, и он предоставит мне более 12 легионов ангелов>>.
\vs p038 6:2 Двенадцать легионов ангелов составляют воинство, насчитывающее 2\,985\,984 пары, или 5\,971\,968 индивидуумов, и 12 таких воинств (35\,831\,808 пар, или 71\,663\,616 индивидуумов) образуют самую большую действующую организацию серафимов~--- ангельскую армию. Серафическое воинство находится под командованием архангела или какой\hyp{}либо другой личности равного статуса, в то время как ангельскими армиями управляют Блистательные Вечерние Звёзды или другие непосредственные заместители Гавриила. А Гавриил является <<верховным главнокомандующим армий небес>>, главой исполнительной власти Властелина Небадона, <<Господа Бога воинств>>\fnst{В древнееврейском тексте Ветхого Завета это обозначение соответствует \textheb{צְבָאוֹת אֱלֹהֵי יהוה} и в синодальном переводе Библии слово <<воинств>> (цева\'от) обычно транслитерируется: <<Господь Бог Саваоф>>, 2\,Царств~5:10.}.
\vs p038 6:3 Хотя они служат под прямым надзором Бесконечного Духа, персонализированного на Спасограде, после посвящения Михаила на Урантии серафимы и все другие категории локальной вселенной подчиняются суверенной власти Сына Властелина. Даже когда Михаил родился во плоти на Урантии, вышла сверхвселенская трансляция по всему Небадону, которая провозглашала: <<И пусть все ангелы поклоняются ему>>\fnst{Евреям~1:6.}. Все категории ангелов подчинены его полновластию; они являются частью той группы, которая была названа <<его могучими ангелами>>\fnst{2\,Фессалоникийцам~1:7.}.
\usection{ХЕРУВИМЫ И САНОВИМЫ}
\vs p038 7:1 По всем существенным дарованиям херувимы и сановимы похожи на серафимов. У них одинаковое происхождение, но не всегда одно и то же предназначение. Они удивительно разумны, необычайно эффективны, трогательно нежны и почти подобны людям. Они являются низшей категорией ангелов и, следовательно, особенно родственны более развитым типам человеческих существ на эволюционных мирах.
\vs p038 7:2 Херувимы и сановимы по своей сути взаимосвязаны, функционально объединены. Один из них является энергетически положительной личностью; другой~--- энергетически отрицательной. Правый дефлектор, или положительно заряженный ангел,~--- это херувим~-- старшая, или управляющая, личность. Левый дефлектор, или отрицательно заряженный ангел,~--- это сановим~--- дополняющее существо. Каждый тип ангела функционально очень ограничен, когда действует в одиночку; поэтому они обычно служат парами. При работе независимо от своих серафических руководителей, они более чем когда\hyp{}либо зависят от взаимного контакта и всегда функционируют вместе.
\vs p038 7:3 \pc Херувимы и сановимы~--- преданные и эффективные помощники серафических служителей, и все семь категорий серафимов обеспечены такими подчинёнными помощниками. Херувимы и сановимы служат в этом качестве веками, но они не сопровождают серафимов в назначениях за пределы локальной вселенной.
\vs p038 7:4 Херувимы и сановимы выполняют повседневную работу духа на индивидуальных мирах систем. В неличностном назначении и в чрезвычайной ситуации они могут служить вместо серафической пары, но они никогда не функционируют, даже временно, в качестве ангелов, сопровождающих человеческих существ; это исключительная привилегия серафимов.
\vs p038 7:5 \pc Получив назначение на планету, херувимы поступают на локальные курсы обучения, включающие изучение планетарных обычаев и языков. Все духи\hyp{}помощники времени двуязычны: они говорят на языке локальной вселенной своего происхождения и на языке своей родной сверхвселенной. В результате обучения в школах миров они овладевают дополнительными языками. Херувимы и сановимы, так же, как серафимы и все другие категории духовных существ, постоянно занимаются самосовершенствованием. Только такие подчинённые существа регулирования мощи и управления энергией, не способны к развитию; все создания, обладающие актуальным или потенциальным волеизъявлением личности, стремятся к новым достижениям.
\vs p038 7:6 \pc Херувимы и сановимы по своей природе очень близки к моронтийному уровню существования, и они оказываются наиболее эффективными в работе на стыке физической, моронтийной и духовной областей. Эти дети Материнского Духа локальной вселенной характеризуются наличием <<четвёртых созданий>>, так же, как Сервиталы Хавоны и миротворческие комиссии. Каждый четвёртый херувим и каждый четвёртый сановим являются квазиматериальными, весьма определённо напоминающими моронтийный уровень существования.
\vs p038 7:7 Эти ангельские четвёртые создания оказывают большую помощь серафимам в более буквальных фазах их вселенской и планетарной деятельности. Такие моронтийные херувимы также выполняют многие необходимые пограничные задания на моронтийных учебных мирах и в больших количествах назначаются на службу в качестве Моронтийных Спутников. Они являются для моронтийных сфер примерно тем же, чем промежуточные создания~--- для эволюционных планет. На обитаемых мирах эти моронтийные херувимы часто работают во взаимодействии с промежуточными созданиями. Херувимы и промежуточные создания~--- совершенно разные категории существ; у них различное происхождение, но в их природе и функциях обнаруживается большое сходство.
\usection{ЭВОЛЮЦИЯ ХЕРУВИМОВ И САНОВИМОВ}
\vs p038 8:1 Херувимам и сановимам открыты многочисленные пути продвижения по службе, ведущие к повышению статуса, который может быть ещё больше усилен объятием Божественной Служительницы. В отношении эволюционного потенциала существуют три больших класса херувимов и сановимов:
\vs p038 8:2 \li{1.}\bibemph{Кандидаты на восхождение}. Эти существа по своей природе являются кандидатами на серафический статус. Херувимы и сановимы этой категории~--- выдающиеся существа, хотя и не равные серафимам по врождённым дарованиям; но благодаря усердию и опыту для них становится возможным достижение полного серафического статуса.
\vs p038 8:3 \li{2.}\bibemph{Херувимы промежуточной фазы}. Не все херувимы и сановимы равны по потенциалу восхождения, и ангелы этой группы по своей природе являются ограниченными из существ ангельских творений. Большинство из них останутся херувимами и сановимами, хотя более одарённые индивидуумы могут достичь ограниченного серафического служения.
\vs p038 8:4 \li{3.}\bibemph{Моронтийные херувимы}. Эти <<четвёртые создания>> ангельских категорий всегда сохраняют свои квазиматериальные свойства. Они останутся херувимами и сановимами вместе с большинством своих собратьев промежуточной фазы вплоть до завершения фактуализации Верховного Существа.
\vs p038 8:5 \pc В то время как потенциал роста второй и третьей групп до некоторой степени ограничен, кандидаты на восхождение могут достичь высот всеобщего серафического служения. Многие из наиболее опытных херувимов прикрепляются к серафическим хранителям предназначения и, таким образом, оказываются на прямом пути к достижению статуса Учителей Обительских Миров после того, как их покинут вышестоящие серафимы. Хранители предназначения не имеют херувимов и сановимов в качестве помощников, когда их смертные подопечные достигают моронтийной жизни. И когда другие категории эволюционных серафимов получают допуск на Серафимоград и Рай, то, покидая пределы Небадона, они должны оставить своих бывших подчинённых. Такие покинутые херувимы и сановимы обычно обнимаются Вселенским Материнским Духом и таким образом обретают серафический статус, достигая уровня, равнозначного уровню Учителя Обительских Миров.
\vs p038 8:6 Когда херувимы и сановимы, однажды объятые, долгое время служат Учителями Обительских Миров на моронтийных сферах, от низшей до высшей, и когда их корпус на Спасограде переполняется, Яркая Утренняя Звезда призывает этих верных слуг созданий времени предстать перед ним. Принимается присяга трансформации личности; и после этого, группами по 7\,000, эти более подготовленные и старшие херувимы и сановимы вновь обнимаются Вселенским Материнским Духом. Из этого второго объятия они выходят полноценными серафимами. Отныне полный и совершенный путь серафима со всеми его Райскими возможностями открыт для таких возрождённых херувимов и сановимов. Такие ангелы могут назначаться хранителями предназначения к какому\hyp{}нибудь смертному существу, и если смертный подопечный достигает выживания, то они получают право на продвижение к Серафимограду и семи кругам серафических достижений~--- вплоть до Рая и Корпуса Завершения.
\usection{ПРОМЕЖУТОЧНЫЕ СОЗДАНИЯ}
\vs p038 9:1 Существует тройная классификация промежуточных созданий: их правильно относят к восходящим Сынам Бога; фактически они группируются с категориями постоянного гражданства, в то время как функционально считаются духами\hyp{}помощниками времени из-за их близкого и эффективного сотрудничества с ангельскими воинствами в работе служения смертному человеку на индивидуальных мирах пространства.
\vs p038 9:2 Эти уникальные создания появляются на большинстве обитаемых миров и их всегда можно найти на десятичных планетах, или планетах экспериментальной жизни, таких как Урантия. Промежуточные создания бывают двух типов~--- первичные и вторичные, и они появляются с помощью следующих методов:
\vs p038 9:3 \li{1.}\bibemph{Первичные промежуточные создания,} более духовная группа, являются до некоторой степени стандартизированной категорией существ, единообразно происходящих от модифицированного восходящего смертного персонала Планетарных Принцев. Численность первичных промежуточных созданий всегда составляет 50\,000, и ни на одной планете, пользующейся их служением, нет большей группы.
\vs p038 9:4 \li{2.}\bibemph{Вторичные промежуточные создания,} более материальная группа этих созданий, численность которых сильно различается на разных мирах, хотя в среднем составляет около 50\,000. Они по\hyp{}разному происходят от планетарных биологических совершенствователей, Адамов и Ев, или от их непосредственных потомков. Существует не менее 24 различных методов, используемых для производства этих вторичных промежуточных созданий на эволюционных мирах пространства. Метод происхождения этой группы на Урантии был необычным и экстраординарным.
\vs p038 9:5 \pc Ни одна из этих групп не является эволюционной случайностью; обе являются существенными чертами в предопределённых планах вселенских зодчих, и их появление на эволюционных мирах при благоприятном стечении обстоятельств соответствует первоначальным образцам и планам развития руководящих Носителей Жизни.
\vs p038 9:6 Первичные промежуточные создания питаются интеллектуальной и духовной энергией с помощью ангельского метода и одинаковы по интеллектуальному статусу. Семь адъютантов разумо\hyp{}духов не вступают с ними в контакт; только шестой и седьмой~--- дух поклонения и дух мудрости~--- могут служить вторичной группе.
\vs p038 9:7 Вторичные промежуточные создания получают физическую энергию Адамическим методом, духовно включены в контур серафическим, и интеллектуально наделены моронтийным переходным типом разума. Они подразделяются на четыре физических типа, семь духовных категорий и двенадцать уровней интеллектуальной реакции на совместное служение двух последних духов\hyp{}адъютантов и моронтийного разума. Это разнообразие определяет различие в их деятельности и в планетарных назначениях.
\vs p038 9:8 Первичные промежуточные создания больше напоминают ангелов, чем смертных; вторичные категории намного больше похожи на человеческих существ. Одни оказывают неоценимую помощь другим в выполнении их разнообразных планетарных заданий. Первичные помощники могут обеспечивать взаимодействие как с регуляторами моронтийной энергии, так и духовной, и с управляющими контуров разума. Вторичная группа может устанавливать рабочие связи только с физическими регуляторами и манипуляторами материальных контуров. Но поскольку каждая категория промежуточных созданий может установить совершенную синхронность контакта с другой, то любая группа тем самым способна достигать практического использования всего энергетического диапазона, простирающегося от грубой физической мощи материальных миров через переходные фазы вселенских энергий до высших сил реальности духа небесных сфер.
\vs p038 9:9 Пропасть между материальным и духовным мирами в совершенстве преодолевается последовательным взаимодействием смертного человека, вторичного промежуточного создания, первичного промежуточного создания, моронтийного херувима, херувима промежуточной фазы и серафима. В личном опыте отдельного смертного эти различные уровни, несомненно, более или менее объединены, и обретают личный смысл благодаря невидимым и таинственным действиям божественного Настройщика Мыслей.
\vs p038 9:10 \pc На нормальных мирах первичные промежуточные создания служат в качестве корпуса разведки и небесных развлечений при Планетарном Принце, в то время как вторичные помощники продолжают своё сотрудничество с Адамическим режимом, содействуя делу развития планетарной цивилизации. В случае отступничества Планетарного Принца и неудачи Материального Сына, как это произошло на Урантии, промежуточные создания становятся подопечными Властелина Системы и служат под прямым руководством исполняющего обязанности хранителя планеты. Но только на трёх других мирах Сатании эти существа функционируют как одна группа под единым руководством, подобно объединённым промежуточным помощникам Урантии.
\vs p038 9:11 Планетарная работа как первичных, так и вторичных промежуточных созданий различна и разнообразна на многочисленных индивидуальных мирах вселенной, но на нормальных и средних планетах их деятельность сильно отличается от обязанностей, которые занимают их время на изолированных сферах, таких как Урантия.
\vs p038 9:12 Первичные промежуточные создания~--- это планетарные историки, которые со времени прибытия Планетарного Принца до эпохи установления в свете и жизни организуют празднества и планируют изображения планетарной истории для выставок планет на столичных мирах систем.
\vs p038 9:13 \pc Промежуточные создания остаются на обитаемом мире долгое время, но, если они верны своему долгу, то в конечном итоге, несомненно, получат признание за свою очень долгую службу в поддержании суверенитета Сына Создателя; они будут должным образом вознаграждены за своё терпеливое служение материальным смертным на их мире времени и пространства. Рано или поздно все признанные [accredited] промежуточные создания войдут в ряды восходящих Сынов Бога и будут должным образом посвящены в долгое приключение Райского восхождения вместе с теми самыми смертными животного происхождения, своими земными братьями, которых они столь ревностно охраняли и которым так эффективно служили в течение долгого планетарного пребывания.
\vsetoff
\vs p038 9:14 [Представлено Мелхиседеком, действующим по просьбе Главы Серафических Воинств Небадона.]
\quizlink
