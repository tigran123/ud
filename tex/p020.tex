\upaper{20}{РАЙСКИЕ СЫНЫ БОГА}
\uminitoc{НИСХОДЯЩИЕ СЫНЫ БОГА}
\uminitoc{СЫНЫ ПОВЕЛИТЕЛИ}
\uminitoc{СУДЕБНЫЕ ДЕЙСТВИЯ}
\uminitoc{МИССИИ ПОВЕЛИТЕЛЯ}
\uminitoc{ПОСВЯЩЕНИЕ РАЙСКИХ СЫНОВ БОГА}
\uminitoc{ПОСВЯЩЕНИЯ В ОБЛИКЕ СМЕРТНЫХ}
\uminitoc{ТРОИЧНЫЕ СЫНЫ УЧИТЕЛЯ}
\uminitoc{СЛУЖЕНИЕ ДАЙНАЛОВ В ЛОКАЛЬНЫХ ВСЕЛЕННЫХ}
\uminitoc{ПЛАНЕТАРНОЕ СЛУЖЕНИЕ ДАЙНАЛОВ}
\uminitoc{ОБЪЕДИНЁННОЕ СЛУЖЕНИЕ РАЙСКИХ СЫНОВ}
\author{Совершенствователь Мудрости}
\vs p020 0:1 По своим функциям в сверхвселенной Орвонтон, Сыны Бога подразделяются на три основные группы:
\vs p020 0:2 \li{1.}Нисходящие Сыны Бога.
\vs p020 0:3 \li{2.}Восходящие Сыны Бога.
\vs p020 0:4 \li{3.}Тринитизованные Сыны Бога.
\vs p020 0:5 \pc К нисходящим категориям сыновства относятся личности непосредственно и божественно созданные. Восходящие сыны, такие как смертные создания, обретают этот статус через эмпирическое участие в творческом процессе, известном как эволюция. Тринитизованные Сыны~--- это группа смешанного происхождения, которая включает все существа, объятые Райской Троицей, даже не имеющие непосредственно Троичного происхождения.
\usection{НИСХОДЯЩИЕ СЫНЫ БОГА}
\vs p020 1:1 Все нисходящие Сыны Бога имеют высокое и божественное происхождение. Они посвящены нисходящему служению на мирах и системах времени и пространства, чтобы способствовать прогрессу в Райском восхождении низших созданий эволюционного происхождения~--- восходящих сынов Бога. В этих повествованиях из многочисленных категорий нисходящих Сынов будут описаны семь. Те Сыны, которые происходят от Божеств на центральном Острове Света и Жизни, называются \bibemph{Райскими Сынами Бога} и охватывают следующие три категории\fnst{Следующие семь категорий распадаются на три группы: $7=3+3+1$. Название первого члена первой и второй групп выбрано из широко известных имён из древнееврейского текста Ветхого Завета: \bibemph{Михаил} от \textheb{מִיכָאֵל} \bibemph{ми ка эль} <<кто подобен Богу?>> и \bibemph{Мелхиседек} от \textheb{מַלְכִּי־צֶדֶק} \bibemph{малки цедек} <<царь праведности>> (см. ниже). Остальные два имени каждой группы сконструированы из двух компонент: окончаний -el/-al и -dek, указывающих на родство с первым элементом группы и корня, несущего основное лексическое значение. Так, \bibemph{Авонал} происходит, несомненно, от древнееврейского \textheb{עָוֺן} \bibemph{авон} <<беззаконие>>, <<наказание за беззаконие>> или <<бедствия, проистекающие от беззаконий>>, а \bibemph{Дайнал} от древнееврейского \textheb{דֵעָה} \bibemph{деа} <<знание>>, особенно <<знание \bibemph{о} Боге>>. Происхождение \bibemph{Ворондадек} от комбинаций саксонских корней vor- $+$ on- $+$ da- $=$ <<в точности такой, как Отец>> и \bibemph{Ланонандек} от lan- $+$ on- $+$ an- $=$ <<не совсем такой, как [Отец]>> весьма сомнительно.}:
\vs p020 1:2 \li{1.}Сыны Создатели~--- Михаилы.
\vs p020 1:3 \li{2.}Сыны Повелители\fnst{По сути, Авоналы~--- это божественные \bibemph{диктаторы}, несущие мирам справедливые наказания за многочисленные беззакония.}~--- Авоналы.
\vs p020 1:4 \li{3.}Троичные Сыны Учителя~--- Дайналы.
\vs p020 1:5 \pc Остальные четыре категории нисходящего сыновства известны как \bibemph{Сыны Бога Локальных Вселенных:}
\vs p020 1:6 \li{4.}Сыны Мелхиседеки\fnst{В древнееврейском тексте Ветхого Завета слово \textheb{מַלְכִּי־צֶדֶק} встречается лишь дважды (см.\,\bibref[93:9.9]{p093 9:9}): Бытие~14:18 и Псалом~110:4 (109:4 по русской Библии, использующей нумерацию Псалмов согласно Септуагинте). В древнееврейском языке добиблейских времён, также как в классическом арабском, существовали три окончания падежей: -u (именительный), -i (родительный) и -a (винительный). В имени собственном \textheb{מַלְכִּי־צֶדֶק}, означающем <<царь праведности>> (\bibemph{малки\hyp{}цедек} $\Rightarrow$ \bibemph{Мелхиседек}), сохранилось это древнее окончание родительного падежа -i, чем объясняется необычная конструктивная форма слова <<царь>> \textheb{מַלְכִּי} \bibemph{малки} вместо логически следующей из правил, но неверной \textheb{מֶלֶךְ} \bibemph{мелех}.}.
\vs p020 1:7 \li{5.}Сыны Ворондадеки.
\vs p020 1:8 \li{6.}Сыны Ланонандеки.
\vs p020 1:9 \li{7.}Носители Жизни.
\vs p020 1:10 \pc Мелхиседеки представляют собой совместное потомство Сына Создателя локальной вселенной, Созидательного Духа и Отца Мелхиседека. И Ворондадеки, и Ланонандеки были созданы Сыном Создателем и его партнёром~--- Созидательным Духом. Ворондадеки наиболее известны как Всевышние~--- Отцы Созвездий; Ланонандеки~--- как Властелины Систем и Планетарные Принцы. Тройственная категория Носителей Жизни создаётся Сыном Создателем и Созидательным Духом в контакте с одним из трёх От Века Древних соответствующей сверхвселенной. Но природа и деятельность этих Сынов Бога Локальных Вселенных точнее отражены в тех документах, которые относятся к локальным творениям.
\vs p020 1:11 \pc Райские Сыны Бога имеют тройственное происхождение: первичные, или Сыны Создатели, создаются Всеобщим Отцом и Вечным Сыном; вторичные, или Сыны Повелители,~--- дети Вечного Сына и Бесконечного Духа; Троичные Сыны Учителя~--- это потомство Отца, Сына и Духа. С точки зрения служения, поклонения и молитвы Райские Сыны едины; их дух един, и их работа идентична по качеству и завершённости.
\vs p020 1:12 Как Райские категории <<От Века>> проявили себя божественными администраторами, так категории Райских Сынов раскрыли себя как божественные служители~--- создатели, помощники, дарители, судьи, учителя и открыватели истины. Они охватывают вселенную вселенных от берегов вечного Острова до обитаемых миров времени и пространства, исполняя различные виды служения в центральной и сверхвселенных, не раскрытые в этих повествованиях. Они по\hyp{}разному организованы в зависимости от характера и места их служения, но в локальной вселенной и Сыны Повелители, и Сыны Учителя служат под руководством Сына Создателя, возглавляющего это владение.
\vs p020 1:13 По\hyp{}видимому, Сыны Создатели обладают сосредоточенным в их личности духовным даром, который они контролируют и могут посвящать, что и сделал ваш собственный Сын Создатель, когда излил свой дух на всю смертную плоть на Урантии. Каждый Сын Создатель наделён этой духовной силой притяжения в своём собственном владении; он лично осознаёт любое действие и эмоцию каждого нисходящего Сына Бога, служащего в его вселенной. В этом выражается божественное отражение~--- повторение в локальной вселенной той абсолютной духовной силы притяжения Вечного Сына, которая позволяет ему дотянуться до всех своих Райских Сынов, чтобы установить и поддерживать контакт с ними, где бы они не находились во всей вселенной вселенных.
\vs p020 1:14 Райские Сыны Создатели действуют не только как Сыны в нисходящих видах служения и посвящения, но по завершении пути своего посвящения каждый становится вселенским Отцом своего собственного творения, в то время как другие Сыны Бога продолжают служение посвящения и духовного возвышения, предназначенного привести планеты, одну за другой, к добровольному признанию исполненного любви правления Всеобщего Отца, достигая кульминации в посвящении созданий воле Райского Отца и в планетарной верности вселенскому владычеству его Сына Создателя.
\vs p020 1:15 В семикратном\fnst{То есть <<прошедшем семь посвящений>>.} Сыне Создателе Создатель и создание навечно объединены союзом понимания, сочувствия и милосердной связи. Вся категория Михаилов~--- Сынов Создателей~--- настолько уникальна, что природа и род их деятельности будут рассмотрены в следующем документе этой серии, в то время как данное повествование будет в основном касаться двух остальных категорий Райского сыновства: Сынов Повелителей и Троичных Сынов Учителей.
\usection{СЫНЫ ПОВЕЛИТЕЛИ}
\vs p020 2:1 Каждый раз, когда оригинальная и абсолютная концепция существа, сформулированная Вечным Сыном, объединяется с новым и божественным идеалом полного любви служения, задуманного Бесконечным Духом, производится новый и оригинальный Сын Бога~--- Райский Сын Повелитель. Эти Сыны составляют категорию Авоналов, в отличие от категории Михаилов, Сынов Создателей. Хотя они и не создатели в личностном смысле, во всей своей деятельности они тесно связаны с Михаилами. Авоналы являются планетарными служителями и судьями, местными судьями время\hyp{}пространственных сфер~--- всех рас, для всех миров и во всех вселенных.
\vs p020 2:2 У нас есть основания полагать, что общее число Сынов Повелителей в большой вселенной составляет около одного миллиарда. Это самоуправляющаяся категория, руководимая своим верховным советом на Рае, составленным из опытных Авоналов из служб всех вселенных. Но по назначении и прибытии в локальную вселенную они служат под руководством Сына Создателя этой области.
\vs p020 2:3 Авоналы~--- Райские Сыны служения и посвящения отдельным планетам локальных вселенных. Поскольку каждый Сын Авонал~--- исключительная личность, и среди них нет двух одинаковых, их работа индивидуально уникальна в сферах их пребывания, где они зачастую воплощаются в облике смертных, а иногда рождаются от земных матерей на эволюционных мирах.
\vs p020 2:4 \pc В дополнение к своему служению на высших административных уровнях Авоналы выполняют тройную функцию на обитаемых мирах:
\vs p020 2:5 \li{1.}\bibemph{Судебные действия}. Они действуют в конце планетарных диспенсаций\fnst{Планетарных <<Судных периодов>>.}. Со временем десятки, или даже сотни, подобных миссий могут осуществляться на каждом отдельном мире, и они могут посещать те же самые или другие миры бессчётное количество раз в качестве завершителей диспенсаций, освободителей спящих выживших созданий.
\vs p020 2:6 \li{2.}\bibemph{Миссии повелителя}. Планетарное посещение такого типа обычно происходит до прибытия Сына посвящения. В такой миссии Авонал появляется в облике зрелого создания данного мира с помощью техники воплощения, не связанной с рождением смертных. После этого первого и обычного визита в качестве повелителя Авоналы могут неоднократно служить повелителем на одной и той же планете как до, так и после появления Сына посвящения. В течение этих дополнительных миссий повелителя Авонал может появиться или не появиться в материальной и видимой форме, но ни в одной из них он не рождается в мир беспомощным младенцем.
\vs p020 2:7 \li{3.}\bibemph{Миссии посвящения}. Все Сыны Авоналы хотя бы однажды посвящают себя одной из смертных рас какого\hyp{}нибудь эволюционного мира. Судебные визиты многочисленны, миссии повелителя могут повторяться, но на каждой планете появляется только один Сын посвящения. Авоналы посвящения рождаются от женщины, как на Урантии воплотился Михаил Небадона.
\vs p020 2:8 \pc Нет предела для числа миссий повелителя и миссий посвящения, в которых могут служить Сыны Авоналы, но обычно, после семикратного повторения опыта, они уступают своё место тем, у кого меньше опыта в таком виде служения. Эти Сыны, получившие опыт многократных пришествий, назначаются в высший личный совет Сына Создателя, таким образом становясь участниками управления делами вселенной.
\vs p020 2:9 Во всей своей работе на обитаемых мирах и для них Сыны Повелители получают помощь от двух категорий существ локальной вселенной: Мелхиседеков и архангелов, а в миссиях посвящения их также сопровождают Блистательные Вечерние Звёзды, тоже берущие начало в локальных творениях. В любых планетарных начинаниях вторичные Райские Сыны, Авоналы, поддерживаются всем могуществом и властью первичного Райского Сына, Сына Создателя локальной вселенной их служения. По сути, их работа на обитаемых сферах столь же эффективна и приемлема, как и служение самог\'о Сына Создателя на таких мирах обитания смертных.
\usection{СУДЕБНЫЕ ДЕЙСТВИЯ}
\vs p020 3:1 
\vs p020 3:2 
\vs p020 3:3 
\vs p020 3:4 
\usection{МИССИИ ПОВЕЛИТЕЛЯ}
\vs p020 4:1 
\vs p020 4:2 
\vs p020 4:3 
\vs p020 4:4 \pc 
\vs p020 4:5 
\usection{ПОСВЯЩЕНИЕ РАЙСКИХ СЫНОВ БОГА}
\vs p020 5:1 
\vs p020 5:2 
\vs p020 5:3 
\vs p020 5:4 
\vs p020 5:5 \pc 
\vs p020 5:6 
\vs p020 5:7 \pc 
\usection{ПОСВЯЩЕНИЯ В ОБЛИКЕ СМЕРТНЫХ}
\vs p020 6:1 
\vs p020 6:2 \pc 
\vs p020 6:3 
\vs p020 6:4 
\vs p020 6:5 \pc 
\vs p020 6:6 \pc 
\vs p020 6:7 
\vs p020 6:8 
\vs p020 6:9 
\usection{ТРОИЧНЫЕ СЫНЫ УЧИТЕЛЯ}
\vs p020 7:1 
\vs p020 7:2 
\vs p020 7:3 
\vs p020 7:4 
\vs p020 7:5 
\usection{СЛУЖЕНИЕ ДАЙНАЛОВ В ЛОКАЛЬНЫХ ВСЕЛЕННЫХ}
\vs p020 8:1 
\vs p020 8:2 
\vs p020 8:3 
\vs p020 8:4 
\usection{ПЛАНЕТАРНОЕ СЛУЖЕНИЕ ДАЙНАЛОВ}
\vs p020 9:1 
\vs p020 9:2 
\vs p020 9:3 
\vs p020 9:4 
\vs p020 9:5 \pc 
\usection{ОБЪЕДИНЁННОЕ СЛУЖЕНИЕ РАЙСКИХ СЫНОВ}
\vs p020 10:1 
\vs p020 10:2 
\vs p020 10:3 
\vs p020 10:4 
\vsetoff
\vs p020 10:5 [Представлено Совершенствователем Мудрости из Уверсы.]
\quizlink
