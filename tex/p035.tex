\upaper{35}{СЫНЫ БОГА ЛОКАЛЬНОЙ ВСЕЛЕННОЙ}
\uminitoc{ОТЕЦ МЕЛХИСЕДЕК}
\uminitoc{СЫНЫ МЕЛХИСЕДЕКИ}
\uminitoc{МИРЫ МЕЛХИСЕДЕКОВ}
\uminitoc{ОСОБАЯ РАБОТА МЕЛХИСЕДЕКОВ}
\uminitoc{СЫНЫ ВОРОНДАДЕКИ}
\uminitoc{ОТЦЫ СОЗВЕЗДИЙ}
\uminitoc{МИРЫ ВОРОНДАДЕКОВ}
\uminitoc{СЫНЫ ЛАНОНАНДЕКИ}
\uminitoc{ЛАНОНАНДЕКИ ПРАВИТЕЛИ}
\uminitoc{МИРЫ ЛАНОНАНДЕКОВ}
\author{Глава Архангелов}
\vs p035 0:1 Представленные ранее Сыны Бога имели Райское происхождение. Они~--- потомки божественных Правителей вселенских владений. Из первой Райской категории сыновства, Сынов Создателей, в Небадоне имеется только один~--- Михаил, вселенский отец и властелин. Из второй категории Райского сыновства, Авоналов или Сынов Повелителей, Небадон обладает полной квотой, в количестве~--- 1\,062. И эти <<меньшие Христы>> так же эффективны и всемогущи в своих планетарных посвящениях, каким был Сын Создатель и Властелин на Урантии. Третья категория, Троичного происхождения, не регистрируется в локальной вселенной, но, по моим оценкам, в Небадоне насчитывается от 15\,000 до 20\,000 Троичных Сынов Учителей, не считая 9\,642 зарегистрированных помощников, тринитизованных созданиями. Эти райские Дайналы не являются ни судьями, ни администраторами; они~--- сверхучителя.
\vs p035 0:2 Типы Сынов, которые будут рассмотрены далее, происходят из локальной вселенной; это~--- потомство Райского Сына Создателя в разнообразных ассоциациях с дополняющим Вселенским Материнским Духом. В этих повествованиях упоминаются следующие категории сыновства локальной вселенной:
\vs p035 0:3 \li{1.}Сыны Мелхиседеки.
\vs p035 0:4 \li{2.}Сыны Воронд\'адеки.
\vs p035 0:5 \li{3.}Сыны Ланон\'андеки.
\vs p035 0:6 \li{4.}Сыны Носители Жизни.
\vs p035 0:7 \pc Действием Триединого Райского Божества создаются три категории сыновства: Михаилы, Авоналы и Дайналы. Двойственное Божество в локальной вселенной, Сын и Дух, также функционирует в создании трёх высоких категорий Сынов: Мелхиседеков, Ворондадеков и Ланонандеков; и, достигнув этого троичного выражения, они сотрудничают со следующим уровнем Бога Семичастного в создании разносторонней категории Носителей Жизни. Эти существа классифицируются вместе с нисходящими Сынами Бога, но они представляют собой уникальную и оригинальную форму вселенской жизни, рассмотрению которой посвящён весь следующий документ.
\usection{ОТЕЦ МЕЛХИСЕДЕК}
\vs p035 1:1 После создания существ, являющихся личными помощниками, таких как Яркая и Утренняя Звезда, и других управляющих личностей, в соответствии с божественным замыслом и созидательными планами данной вселенной, возникает новая форма созидательного союза между Сыном Создателем и Созидательным Духом, Дочерью Бесконечного Духа в локальной вселенной. Личностный потомок, появившийся в результате этого творческого партнёрства, это изначальный Мелхиседек~--- Отец Мелхиседек~--- то уникальное существо, которое впоследствии сотрудничает с Сыном Создателем и Созидательным Духом, производя на свет всю одноимённую группу.
\vs p035 1:2 Во вселенной Небадон Отец Мелхиседек действует как первый исполнительный помощник Яркой и Утренней Звезды. Гавриил больше занят вселенским планированием. Мелхиседек~--- практическими процедурами. Гавриил возглавляет регулярно собираемые суды и советы Небадона, Мелхиседек~--- специальные, внеочередные и чрезвычайные комиссии и совещательные органы. Гавриил и Отец Мелхиседек никогда не покидают Спасоград одновременно\fnst{Иногда они всё же покидают Спасоград одновременно, см.\,\bibref[158:1.6--7]{p158 1:6}.}, так как в отсутствие Гавриила Отец Мелхиседек выполняет функции главы Небадона.
\vs p035 1:3 Все Мелхиседеки нашей вселенной были созданы в течение одного тысячелетия стандартного времени Сыном Создателем и Созидательным Духом в союзе с Отцом Мелхиседеком. Будучи категорией сыновства, в которой один из них действовал как равноправный создатель, по своему строению Мелхиседеки отчасти происходят от самих себя и поэтому являются кандидатами на реализацию высшей формы самоуправления. Они периодически выбирают своего собственного административного руководителя сроком на семь лет стандартного времени и в остальном действуют как саморегулирующаяся категория, хотя изначальный Мелхиседек пользуется некоторыми неотъемлемыми прерогативами совместного родителя. Время от времени этот Отец Мелхиседек назначает определённых индивидуумов своей категории в качестве особых Носителей Жизни на мидсонитные миры~--- обитаемые планеты не раскрытого пока ещё на Урантии типа.
\vs p035 1:4 Мелхиседеки не ведут широкой деятельности за пределами локальной вселенной, за исключением тех случаев, когда их вызывают в качестве свидетелей по делам, рассматриваемым судами сверхвселенной, и когда они назначаются, как это иногда бывает, специальными послами для представления одной вселенной перед другой в пределах своей сверхвселенной. Изначальный или первородный Мелхиседек каждой вселенной всегда волен отправиться в соседние вселенные или в Рай с миссиями, связанными с интересами и обязанностями своей категории.
\usection{СЫНЫ МЕЛХИСЕДЕКИ}
\vs p035 2:1 
\vs p035 2:2 
\vs p035 2:3 \pc 
\vs p035 2:4 
\vs p035 2:5 
\vs p035 2:6 
\vs p035 2:7 
\vs p035 2:8 \pc 
\vs p035 2:9 
\usection{МИРЫ МЕЛХИСЕДЕКОВ}
\vs p035 3:1 
\vs p035 3:2 
\vs p035 3:3 
\vs p035 3:4 
\vs p035 3:5 
\vs p035 3:6 
\vs p035 3:7 
\vs p035 3:8 
\vs p035 3:9 
\vs p035 3:10 \pc 
\vs p035 3:11 \pc 
\vs p035 3:12 
\vs p035 3:13 \pc 
\vs p035 3:14 
\vs p035 3:15 
\vs p035 3:16 
\vs p035 3:17 
\vs p035 3:18 
\vs p035 3:19 
\vs p035 3:20 \pc 
\vs p035 3:21 
\vs p035 3:22 
\usection{ОСОБАЯ РАБОТА МЕЛХИСЕДЕКОВ}
\vs p035 4:1 
\vs p035 4:2 \pc 
\vs p035 4:3 
\vs p035 4:4 
\vs p035 4:5 \pc 
\usection{СЫНЫ ВОРОНДАДЕКИ}
\vs p035 5:1 
\vs p035 5:2 \pc 
\vs p035 5:3 \pc 
\vs p035 5:4 
\vs p035 5:5 
\vs p035 5:6 
\vs p035 5:7 
\usection{ОТЦЫ СОЗВЕЗДИЙ}
\vs p035 6:1 
\vs p035 6:2 
\vs p035 6:3 
\vs p035 6:4 
\vs p035 6:5 
\usection{МИРЫ ВОРОНДАДЕКОВ}
\vs p035 7:1 
\vs p035 7:2 
\vs p035 7:3 
\usection{СЫНЫ ЛАНОНАНДЕКИ}
\vs p035 8:1 
\vs p035 8:2 
\vs p035 8:3 
\vs p035 8:4 
\vs p035 8:5 
\vs p035 8:6 
\vs p035 8:7 \pc 
\vs p035 8:8 
\vs p035 8:9 \pc 
\vs p035 8:10 
\vs p035 8:11 
\vs p035 8:12 
\vs p035 8:13 
\vs p035 8:14 
\vs p035 8:15 \pc 
\usection{ЛАНОНАНДЕКИ ПРАВИТЕЛИ}
\vs p035 9:1 
\vs p035 9:2 
\vs p035 9:3 
\vs p035 9:4 
\vs p035 9:5 \pc 
\vs p035 9:6 
\vs p035 9:7 
\vs p035 9:8 
\vs p035 9:9 \pc 
\vs p035 9:10 
\usection{МИРЫ ЛАНОНАНДЕКОВ}
\vs p035 10:1 
\vs p035 10:2 
\vs p035 10:3 
\vs p035 10:4 \pc 
\vs p035 10:5 \pc 
\vsetoff
\vs p035 10:6 
\quizlink
