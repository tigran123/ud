\upaper{35}{СЫНЫ БОГА ЛОКАЛЬНОЙ ВСЕЛЕННОЙ}
\uminitoc{ОТЕЦ МЕЛХИСЕДЕК}
\uminitoc{СЫНЫ МЕЛХИСЕДЕКИ}
\uminitoc{МИРЫ МЕЛХИСЕДЕКОВ}
\uminitoc{ОСОБАЯ РАБОТА МЕЛХИСЕДЕКОВ}
\uminitoc{СЫНЫ ВОРОНДАДЕКИ}
\uminitoc{ОТЦЫ СОЗВЕЗДИЙ}
\uminitoc{МИРЫ ВОРОНДАДЕКОВ}
\uminitoc{СЫНЫ ЛАНОНАНДЕКИ}
\uminitoc{ЛАНОНАНДЕКИ ПРАВИТЕЛИ}
\uminitoc{МИРЫ ЛАНОНАНДЕКОВ}
\author{Глава Архангелов}
\vs p035 0:1 Представленные ранее Сыны Бога имели Райское происхождение. Они~--- потомки божественных Правителей вселенских владений. Из первой Райской категории сыновства, Сынов Создателей, в Небадоне имеется только один~--- Михаил, вселенский отец и властелин. Из второй категории Райского сыновства, Авоналов или Сынов Повелителей, Небадон обладает полной квотой, в количестве~--- 1\,062. И эти <<меньшие Христы>> так же эффективны и всемогущи в своих планетарных посвящениях, каким был Сын Создатель и Властелин на Урантии. Третья категория, Троичного происхождения, не регистрируется в локальной вселенной, но, по моим оценкам, в Небадоне насчитывается от 15\,000 до 20\,000 Троичных Сынов Учителей, не считая 9\,642 зарегистрированных помощников, тринитизованных созданиями. Эти райские Дайналы не являются ни судьями, ни администраторами; они~--- сверхучителя.
\vs p035 0:2 Типы Сынов, которые будут рассмотрены далее, происходят из локальной вселенной; это~--- потомство Райского Сына Создателя в разнообразных ассоциациях с дополняющим Вселенским Материнским Духом. В этих повествованиях упоминаются следующие категории сыновства локальной вселенной:
\vs p035 0:3 \li{1.}Сыны Мелхиседеки.
\vs p035 0:4 \li{2.}Сыны Ворондадеки.
\vs p035 0:5 \li{3.}Сыны Ланонандеки.
\vs p035 0:6 \li{4.}Сыны Носители Жизни.
\vs p035 0:7 \pc Действием Триединого Райского Божества создаются три категории сыновства: Михаилы, Авоналы и Дайналы. Двойственное Божество в локальной вселенной, Сын и Дух, также функционирует в создании трёх высоких категорий Сынов: Мелхиседеков, Ворондадеков и Ланонандеков; и, достигнув этого троичного выражения, они сотрудничают со следующим уровнем Бога Семичастного в создании разносторонней категории Носителей Жизни. Эти существа классифицируются вместе с нисходящими Сынами Бога, но они представляют собой уникальную и оригинальную форму вселенской жизни, рассмотрению которой посвящён весь следующий документ.
\usection{ОТЕЦ МЕЛХИСЕДЕК}
\vs p035 1:1 После создания существ, являющихся личными помощниками, таких как Яркая Утренняя Звезда, и других управляющих личностей, в соответствии с божественным замыслом и созидательными планами данной вселенной, возникает новая форма созидательного союза между Сыном Создателем и Созидательным Духом, Дочерью Бесконечного Духа в локальной вселенной. Личностный потомок, появившийся в результате этого творческого партнёрства, это изначальный Мелхиседек~--- Отец Мелхиседек~--- то уникальное существо, которое впоследствии сотрудничает с Сыном Создателем и Созидательным Духом, производя на свет всю одноимённую группу.
\vs p035 1:2 Во вселенной Небадон Отец Мелхиседек действует как первый исполнительный помощник Яркой Утренней Звезды. Гавриил больше занят вселенским планированием. Мелхиседек~--- практическими процедурами. Гавриил возглавляет регулярно собираемые суды и советы Небадона, Мелхиседек~--- специальные, внеочередные и чрезвычайные комиссии и совещательные органы. Гавриил и Отец Мелхиседек никогда не покидают Спасоград одновременно\fnst{Иногда они всё же покидают Спасоград одновременно, см.\,\bibref[158:1.6--7]{p158 1:6}.}, так как в отсутствие Гавриила Отец Мелхиседек выполняет функции главы Небадона.
\vs p035 1:3 Все Мелхиседеки нашей вселенной были созданы в течение одного тысячелетия стандартного времени Сыном Создателем и Созидательным Духом в союзе с Отцом Мелхиседеком. Будучи категорией сыновства, в которой один из них действовал как равноправный создатель, по своему строению Мелхиседеки отчасти происходят от самих себя и поэтому являются кандидатами на реализацию высшей формы самоуправления. Они периодически выбирают своего собственного административного руководителя сроком на семь лет стандартного времени и в остальном действуют как саморегулирующаяся категория, хотя изначальный Мелхиседек пользуется некоторыми неотъемлемыми прерогативами совместного родителя. Время от времени этот Отец Мелхиседек назначает определённых индивидуумов своей категории в качестве особых Носителей Жизни на мидсонитные миры~--- обитаемые планеты не раскрытого пока ещё на Урантии типа.
\vs p035 1:4 Мелхиседеки не ведут широкой деятельности за пределами локальной вселенной, за исключением тех случаев, когда их вызывают в качестве свидетелей по делам, рассматриваемым судами сверхвселенной, и когда они назначаются, как это иногда бывает, специальными послами для представления одной вселенной перед другой в пределах своей сверхвселенной. Изначальный или первородный Мелхиседек каждой вселенной всегда волен отправиться в соседние вселенные или в Рай с миссиями, связанными с интересами и обязанностями своей категории.
\usection{СЫНЫ МЕЛХИСЕДЕКИ}
\vs p035 2:1 Мелхиседеки~--- это первая категория божественных Сынов, сто\'ящая достаточно близко к низшим формам созданной жизни, чтобы функционировать непосредственно оказанием помощи по подъёму смертных и служением эволюционным расам без необходимости воплощения. Эти Сыны естественным образом находятся на полпути великого личностного нисхождения, по происхождению занимая примерно среднее положение между высочайшей Божественностью и низшим уровнем жизни созданий, наделённых волей. Это делает их естественными посредниками между высшими, божественными, уровнями живого существования и низшими, даже материальными, формами жизни на эволюционных мирах. Серафические категории, ангелы, с удовольствием работают с Мелхиседеками; фактически все формы разумной жизни находят в этих Сынах понимающих друзей, сочувствующих учителей и мудрых советников.
\vs p035 2:2 Мелхиседеки~--- самоуправляющаяся категория. В этой уникальной группе мы сталкиваемся с первой попыткой самоопределения со стороны существ локальной вселенной и наблюдаем высший тип истинного самоуправления. Эти Сыны организуют собственные средства для управления своей группой и планетой обитания, а также для шести связанных сфер и подчинённых им мирам. И следует отметить, что они никогда не злоупотребляли своими прерогативами; ни разу во всей сверхвселенной Орвонтон эти Сыны Мелхиседеки не обманули оказанного им доверия. Они~--- надежда каждой вселенской группы, стремящейся к самоуправлению, они являются образцом и учителями самоуправления для всех сфер Небадона. Все категории разумных существ, от вышестоящих руководителей до нижестоящих подчинённых, искренне восхваляют правление Мелхиседеков.
\vs p035 2:3 \pc Категория сыновства Мелхиседек занимает положение и берёт на себя ответственность старшего сына в большой семье. Наибольшая часть их работы регулярна и несколько рутинна, но значительная её часть выполняется совершенно добровольно и по собственной инициативе. Большинство специальных ассамблей, которые время от времени собираются на Спасограде, созываются по предложению Мелхиседеков. По своей собственной инициативе эти Сыны патрулируют свою родную вселенную. Они поддерживают автономную организацию, занимающуюся сбором вселенских данных, периодически отчитываясь перед Сыном Создателем, независимо от всей информации, поступающей в столицу вселенной через регулярные службы повседневного управления миром. По своей природе они являются беспристрастными наблюдателями и пользуются полным доверием всех классов разумных существ.
\vs p035 2:4 Мелхиседеки действуют как мобильные и совещательные кассационные суды сфер; эти вселенские Сыны отправляются небольшими группами в миры, чтобы служить в качестве консультативных комиссий, снимать показания, получать предложения и действовать как советники, тем самым помогая преодолевать основные трудности и улаживать серьёзные разногласия, которые возникают время от времени в делах эволюционных владений.
\vs p035 2:5 Эти старшие Сыны вселенной~--- главные помощники Яркой Утренней Звезды в выполнении мандатов Сына Создателя. Когда Мелхиседек отправляется в отдалённый мир от имени Гавриила, то для целей этой конкретной миссии он может представлять своего отправителя и в таком случае он появится на планете назначения наделённый всей полнотой власти Яркой Утренней Звезды. Особенно это относится к тем сферам, где более высокий Сын ещё не воплощался в облике создания данного мира.
\vs p035 2:6 Когда Сын Создатель вступает на путь посвящения эволюционному миру, он отправляется туда один; но когда к посвящению приступает один из его Райских братьев, Сын Авонал, его сопровождают 12 Мелхиседеков поддержки, которые столь эффективно способствуют успеху миссии посвящения. Они также поддерживают Райских Авоналов, посещающих обитаемые миры с миссиями повеления, и при выполнении этих заданий Мелхиседеки видимы глазам смертных, если Сын Авонал проявляет себя таким же образом.
\vs p035 2:7 Нет ни одной фазы планетарной духовной потребности, в которой бы они не служили. Это те учителя, которые так часто побуждают целые миры высокоразвитой жизни прийти к окончательному и полному признанию ими Сына Создателя и его Райского Отца.
\vs p035 2:8 \pc Мелхиседеки почти совершенны в мудрости, но не безошибочны в суждениях. Находясь в изоляции и одиночестве при выполнении планетарной миссии, они иногда ошибались в незначительных вопросах, то есть выбранный ими определённый образ действий впоследствии не одобрялся их руководителями. Такая ошибка суждения временно дисквалифицирует Мелхиседека до тех пор, пока он не отправится на Спасоград и, на аудиенции у Сына Создателя, не получит наставление, эффективно устраняющее ту дисгармонию, которая привела к разногласиям с его собратьями; а на третий день, после исправительного отдыха, его восстанавливают на службе. Но эти незначительные промахи в действиях Мелхиседеков редко случались в Небадоне.
\vs p035 2:9 Категория этих Сынов не возрастает в количестве; их число постоянно, хотя и варьируется в каждой локальной вселенной. Число Мелхиседеков, зарегистрированных на их столичной планете в Небадоне, превышает 10\,000\,000.
\usection{МИРЫ МЕЛХИСЕДЕКОВ}
\vs p035 3:1 
\vs p035 3:2 
\vs p035 3:3 
\vs p035 3:4 
\vs p035 3:5 
\vs p035 3:6 
\vs p035 3:7 
\vs p035 3:8 
\vs p035 3:9 
\vs p035 3:10 \pc 
\vs p035 3:11 \pc 
\vs p035 3:12 
\vs p035 3:13 \pc 
\vs p035 3:14 
\vs p035 3:15 
\vs p035 3:16 
\vs p035 3:17 
\vs p035 3:18 
\vs p035 3:19 
\vs p035 3:20 \pc 
\vs p035 3:21 
\vs p035 3:22 
\usection{ОСОБАЯ РАБОТА МЕЛХИСЕДЕКОВ}
\vs p035 4:1 
\vs p035 4:2 \pc 
\vs p035 4:3 
\vs p035 4:4 
\vs p035 4:5 \pc 
\usection{СЫНЫ ВОРОНДАДЕКИ}
\vs p035 5:1 
\vs p035 5:2 \pc 
\vs p035 5:3 \pc 
\vs p035 5:4 
\vs p035 5:5 
\vs p035 5:6 
\vs p035 5:7 
\usection{ОТЦЫ СОЗВЕЗДИЙ}
\vs p035 6:1 
\vs p035 6:2 
\vs p035 6:3 
\vs p035 6:4 
\vs p035 6:5 
\usection{МИРЫ ВОРОНДАДЕКОВ}
\vs p035 7:1 
\vs p035 7:2 
\vs p035 7:3 
\usection{СЫНЫ ЛАНОНАНДЕКИ}
\vs p035 8:1 
\vs p035 8:2 
\vs p035 8:3 
\vs p035 8:4 
\vs p035 8:5 
\vs p035 8:6 
\vs p035 8:7 \pc 
\vs p035 8:8 
\vs p035 8:9 \pc 
\vs p035 8:10 
\vs p035 8:11 
\vs p035 8:12 
\vs p035 8:13 
\vs p035 8:14 
\vs p035 8:15 \pc 
\usection{ЛАНОНАНДЕКИ ПРАВИТЕЛИ}
\vs p035 9:1 
\vs p035 9:2 
\vs p035 9:3 
\vs p035 9:4 
\vs p035 9:5 \pc 
\vs p035 9:6 
\vs p035 9:7 
\vs p035 9:8 
\vs p035 9:9 \pc 
\vs p035 9:10 
\usection{МИРЫ ЛАНОНАНДЕКОВ}
\vs p035 10:1 
\vs p035 10:2 
\vs p035 10:3 
\vs p035 10:4 \pc 
\vs p035 10:5 \pc 
\vsetoff
\vs p035 10:6 
\quizlink
