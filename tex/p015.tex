\upaper{15}{СЕМЬ СВЕРХВСЕЛЕННЫХ}
\uminitoc{ПРОСТРАНСТВЕННЫЙ УРОВЕНЬ СВЕРХВСЕЛЕННЫХ}
\uminitoc{ОРГАНИЗАЦИЯ СВЕРХВСЕЛЕННЫХ}
\uminitoc{СВЕРХВСЕЛЕННАЯ ОРВОНТОН}
\uminitoc{ТУМАННОСТИ --- ПРАРОДИТЕЛИ ВСЕЛЕННЫХ}
\uminitoc{ПРОИСХОЖДЕНИЕ ТЕЛ В ПРОСТРАНСТВЕ}
\uminitoc{СФЕРЫ ПРОСТРАНСТВА}
\uminitoc{АРХИТЕКТУРНЫЕ СФЕРЫ}
\uminitoc{КОНТРОЛЬ И РЕГУЛИРОВАНИЕ ЭНЕРГИИ}
\uminitoc{КОНТУРЫ СВЕРХВСЕЛЕННЫХ}
\uminitoc{ПРАВИТЕЛИ СВЕРХВСЕЛЕННЫХ}
\uminitoc{СОВЕЩАТЕЛЬНАЯ АССАМБЛЕЯ}
\uminitoc{ВЕРХОВНЫЕ СУДЫ}
\uminitoc{ПРАВИТЕЛЬСТВА СЕКТОРОВ}
\uminitoc{ЦЕЛИ СЕМИ СВЕРХВСЕЛЕННЫХ}
\author{Всеобщий Цензор}
\vs p015 0:1 Для Всеобщего Отца --- как для Отца --- вселенные фактически не существуют; он имеет дело с личностями; он является Отцом личностей. Для Вечного Сына и Бесконечного Духа --- как для создателей\hyp{}партнёров --- вселенные представляют собой локализованные и индивидуальные творения, находящиеся под совместным правлением Сынов Создателей и Созидательных Духов. Для Райской Троицы за пределами Хавоны существует всего семь обитаемых вселенных, --- семь сверхвселенных, в юрисдикцию которых входит круг первого уровня пост\hyp{}Хавонского пространства. Семь Главных Духов излучают своё влияние из центрального Острова, делая тем самым огромное творение одним гигантским колесом, центральный стержень которого --- вечный Остров Рай, семь спиц --- излучения Семи Главных Духов, обод --- внешние области большой вселенной.
\vs p015 0:2 В начале материализации всеобщего творения была сформулирована семеричная схема организации и правления сверхвселенными. Первое творение после Хавоны было разделено на семь колоссальных сегментов, и были спроектированы и сконструированы столичные миры для правительств этих сверхвселенных. Нынешняя система управления существует почти от вечности, и правители этих семи сверхвселенных справедливо называются От Века Древними.
\vs p015 0:3 Из огромного объёма знаний о сверхвселенных я могу поведать тебе лишь немногое, но повсюду в этих областях действует метод разумного управления как над физическими, так и над духовными силами, а всеобщие гравитационные присутствия функционируют там с величественной мощью и в совершенной гармонии. Вначале важно получить адекватное представление о физическом устройстве и материальной организации областей сверхвселенных, поскольку тогда ты сможешь лучше усвоить значение изумительной организации, предусмотренной для их духовного управления и интеллектуального развития волевых созданий, живущих на мириадах обитаемых планет, разбросанных по этим семи сверхвселенным.
\usection{ПРОСТРАНСТВЕННЫЙ УРОВЕНЬ СВЕРХВСЕЛЕННЫХ}
\vs p015 1:1 В записях, наблюдениях и воспоминаниях, ограниченных миллионом или миллиардом ваших коротких лет, Урантия и вселенная, к которой она принадлежит, участвуют в приключении долгого и неизведанного устремления в новое пространство; но согласно записям Уверсы, в соответствии с более ранними наблюдениями и в гармонии с более обширным опытом и расчётами нашего уровня, и в результате заключений, основанных на этих и других выводах, мы знаем, что вселенные вовлечены в упорядоченную, хорошо осмысленную и безупречно управляемую процессию, в величественном великолепии кружащуюся вокруг Первого Великого Источника и Центра и вселенной его пребывания.
\vs p015 1:2 Мы давно обнаружили, что семь сверхвселенных обращаются по громадному эллипсу, гигантскому и вытянутому кругу. Ваша солнечная система и другие миры времени отнюдь не устремляются в неизведанное пространство неуправляемо, без карты и компаса. Локальная вселенная, к которой принадлежит ваша система, следует определённым и хорошо осмысленным курсом, двигаясь против часовой стрелки по огромной замкнутой дуге вокруг центральной вселенной. Этот космический путь полностью нанесён на карты и так же хорошо известен сверхвселенским наблюдателям звёзд, как орбиты планет, составляющих вашу солнечную систему, известны астрономам Урантии.
\vs p015 1:3 Урантия расположена в ещё не полностью сформированных как локальной, так и сверхвселенной, и ваша локальная вселенная находится в непосредственной близости от многочисленных частично завершённых физических творений. Вы принадлежите к одной из относительно молодых вселенных. Однако сегодня вы не устремляетесь наугад в неизведанное пространство и не качаетесь слепо в неизвестных регионах. Вы следуете упорядоченным и предусмотренным путём, относящимся к пространственному уровню сверхвселенных. Сейчас вы проходите через то же пространство, которое ваша планетарная система или её предшественники пересекали много веков назад; и когда\hyp{}нибудь в далёком будущем ваша система или её преемники вновь пересекут то же самое пространство, сквозь которое вы сейчас столь стремительно проноситесь.
\vs p015 1:4 \pc В данную эпоху и согласно принятым на Урантии представлениям о направлении, сверхвселенная номер один поворачивается почти строго на север, к востоку от Райской резиденции Великих Источников и Центров и центральной вселенной Хавоны, приблизительно напротив неё. Эта позиция, вместе с соответствующей ей на западе, представляет собой ближайшее физическое приближение сфер времени к вечному Острову. Сверхвселенная номер два расположена на севере, готовясь к повороту в западном направлении, а третья сейчас занимает самый северный сегмент великого пространственного пути, уже повернув в изгиб, ведущий к южному направлению. Номер четыре проходит сравнительно прямой участок полёта в южном направлении, и её передние области теперь приближаются к позиции напротив Великих Центров. Номер пять почти покинула свою позицию напротив Центра Центров, продолжая двигаться прямым курсом на юг, как раз перед поворотом на восток; номер шесть занимает б\'ольшую часть южной дуги, сегмент которой ваша сверхвселенная уже почти прошла.
\vs p015 1:5 Ваша локальная вселенная Небадон принадлежит к Орвонтону --- седьмой сверхвселенной, обращающейся между первой и шестой сверхвселенными, не так давно обогнувшей (согласно нашим представлениям о времени) юго\hyp{}восточный изгиб пространственного уровня сверхвселенных. Сегодня солнечная система, к которой принадлежит Урантия, прошла за несколько предыдущих миллиардов лет поворот вокруг южного изгиба, так что вы только сейчас покидаете юго\hyp{}восточный изгиб и быстро продвигаетесь по длинному и относительно прямому пути в северном направлении. В течение бессчётных веков Орвонтон будет следовать этим почти прямым курсом на север.
\vs p015 1:6 Урантия принадлежит к системе, расположенной недалеко от границы вашей локальной вселенной; а ваша локальная вселенная в настоящее время пересекает периферию Орвонтона. За вами есть ещё другие творения, но вы сильно удалены в пространстве от тех физических систем, которые обращаются по великому кругу в сравнительной близости к Великому Источнику и Центру.
\usection{ОРГАНИЗАЦИЯ СВЕРХВСЕЛЕННЫХ}
\vs p015 2:1 Только Всеобщий Отец знает местонахождение и фактическое число обитаемых миров в пространстве; он называет их всех по имени и номеру. Я могу назвать лишь приблизительное число обитаемых или пригодных для обитания планет, ибо в одних локальных вселенных больше миров, пригодных для разумной жизни, чем в других. Кроме того, ещё не все спроектированные локальные вселенные организованы. Поэтому предлагаемые мной оценки предназначены исключительно для того, чтобы дать некоторое представление о необъятности материального творения.
\vs p015 2:2 \pc Большая вселенная включает семь сверхвселенных, и устроены они приблизительно так:
\vs p015 2:3 \li{1.}\bibemph{Система}. Основная единица сверхуправления состоит примерно из 1000 обитаемых или пригодных для обитания миров. Пылающие солнца, холодные миры, планеты, слишком близко расположенные к горячим солнцам, и другие сферы, не пригодные для обитания созданий, не включены в эту группу. Эти 1000 миров, приспособленных для поддержания жизни, называются системой, однако в более молодых системах лишь сравнительно небольшое количество миров может быть обитаемым. Каждую обитаемую планету возглавляет Планетарный Принц, а каждая локальная система имеет в качестве своей столицы архитектурную сферу, управляемую Властелином Системы.
\vs p015 2:4 \li{2.}\bibemph{Созвездие}. Сто систем (около 100\,000 пригодных для обитания планет) составляют созвездие. Каждое созвездие имеет столичную архитектурную сферу и возглавляется тремя Сынами Ворондадеками, Всевышними. Каждое созвездие находится также под наблюдением От Века Верного, посла Райской Троицы.
\vs p015 2:5 \li{3.}\bibemph{Локальная вселенная}. Сто созвездий (около 10\,000\,000 пригодных для обитания планет) составляют локальную вселенную. Каждая локальная вселенная имеет великолепный архитектурный столичный мир и управляется одним из равных Сынов Создателей Бога категории Михаил. Каждая вселенная благословлена присутствием От Века Единого, представителя Райской Троицы.
\vs p015 2:6 \li{4.}\bibemph{Малый сектор}. Сто локальных вселенных (около 1\,000\,000\,000 пригодных для обитания планет) составляют малый сектор управления сверхвселенной; он имеет прекрасный столичный мир, откуда его правители, От Века Недавние, руководят делами малого сектора. На столичном центре каждого малого сектора находятся трое От Века Недавних, Верховных Троичных Личностей.
\vs p015 2:7 \li{5.}\bibemph{Большой сектор}. Сто малых секторов (около 100\,000\,000\,000 пригодных для обитания миров) составляют один большой сектор. Каждый большой сектор снабжён превосходным центром и возглавляется тремя От Века Совершенными, Верховными Троичными Личностями.
\vs p015 2:8 \li{6.}\bibemph{Сверхвселенная}. Десять больших секторов (около 1\,000\,000\,000\,000 пригодных для обитания планет) составляют сверхвселенную. Каждая сверхвселенная обеспечена огромным и великолепным столичным миром и управляется тремя От Века Древними.
\vs p015 2:9 \li{7.}\bibemph{Большая Вселенная}. Семь сверхвселенных составляют нынешнюю организованную большую вселенную, состоящую примерно из семи триллионов пригодных для обитания миров, а также архитектурных сфер\fnst{Каждая локальная вселенная содержит в точности 647\,591 архитектурную сферу (см.\,\bibref[37:10.1]{p037 10:1}), а семь сверхвселенных, соответственно, 700\,000$\times$647\,591=453\,313\,700\,000 архитектурных сфер, то есть в организации вселенных предусмотрено примерно 15 обитаемых миров на один архитектурный.} и одного миллиарда обитаемых сфер Хавоны. Сверхвселенными опосредованно и с помощью системы отражения управляют и руководят из Рая Семь Главных Духов. Миллиардом миров Хавоны непосредственно управляют От Века Вечные, одна из таких Верховных Троичных Личностей возглавляет каждую из этих совершенных сфер.
\vs p015 2:10 \pc За исключением сфер системы Рай\hyp{}Хавона, план организации вселенных предусматривает следующие единицы:
\vs p015 2:11 сверхвселенные\hfill7
\vs p015 2:12 большие сектора\hfill70
\vs p015 2:13 малые сектора\hfill7\,000
\vs p015 2:14 локальные вселенные\hfill700\,000
\vs p015 2:15 созвездия\hfill70\,000\,000
\vs p015 2:16 локальные системы\hfill7\,000\,000\,000
\vs p015 2:17 пригодные для обитания планеты\hfill7\,000\,000\,000\,000.
\vs p015 2:18 Состав каждой из семи сверхвселенных выглядит примерно так:
\vs p015 2:19 одна система включает примерно\hfill1\,000 миров
\vs p015 2:20 одно созвездие (100 систем)\hfill100\,000 миров
\vs p015 2:21 одна вселенная (100 созвездий)\hfill10\,000\,000 миров
\vs p015 2:22 один малый сектор (100 вселенных)\hfill1\,000\,000\,000 миров
\vs p015 2:23 один большой сектор (100 малых секторов)\hfill100\,000\,000\,000 миров
\vs p015 2:24 одна сверхвселенная (10 больших секторов)\hfill1\,000\,000\,000\,000 миров.
\vs p015 2:25 \pc Все подобные оценки в лучшем случае приблизительны, ибо постоянно развиваются новые системы, в то время как другие формирования временно выбывают из материального существования.
\usection{СВЕРХВСЕЛЕННАЯ ОРВОНТОН}
\vs p015 3:1 Практически все звёздные миры, видимые невооружённым глазом на Урантии, принадлежат седьмой части большой вселенной --- сверхвселенной Орвонтон. Обширная звёздная система Млечный Путь представляет собой центральное ядро Орвонтона и большей частью находится за пределами вашей локальной вселенной. Это огромное скопление солнц, тёмных островов пространства, двойных звёзд, шаровых скоплений, звёздных облаков, спиральных и иных туманностей\fnst{В начале XX века галактики назывались <<спиральными туманностями>>. См.\,\bibref[12:1.12]{p012 1:12}, \bibref[12:2.2]{p012 2:2}.} вместе с мириадами отдельных планет образует похожую на часы удлинённо\hyp{}круглую группу, составляющую примерно одну седьмую обитаемых эволюционных вселенных.
\vs p015 3:2 Если смотреть из астрономического положения Урантии в направлении великого Млечного Пути через поперечное сечение соседних систем, видно, что сферы Орвонтона движутся в огромной вытянутой плоскости, ширина которой намного больше толщины, а длина значительно превышает ширину.
\vs p015 3:3 Наблюдение так называемого Млечного Пути обнаруживает сравнительное увеличение концентрации звёзд Орвонтона при обозрении неба в определённом направлении, тогда как по обе стороны концентрация уменьшается; число звёзд и других сфер уменьшается по мере удаления от главной плоскости нашей материальной сверхвселенной. При благоприятном угле наблюдения сквозь основное тело этой области максимальной концентрации, вы смотрите в направлении вселенной\hyp{}резиденции и центра всех вещей.
\vs p015 3:4 \pc Из десяти больших подразделений Орвонтона восемь приблизительно идентифицированы астрономами Урантии. Два других трудно распознать по отдельности, потому что вы вынуждены рассматривать эти явления изнутри. Если бы вы могли взглянуть на сверхвселенную Орвонтон с очень дальнего расстояния в пространстве, вы бы сразу узнали десять больших секторов седьмой галактики.
\vs p015 3:5 Центр вращения вашего малого сектора расположен далеко в огромном и плотном звёздном облаке Стрельца, вокруг которого движется ваша локальная вселенная и связанные с ней творения, и с противоположных сторон обширной субгалактической системы Стрельца вы можете наблюдать два великих потока звёздных облаков, возникающих в виде колоссальных звёздных витков\fnst{Или <<звёздных колец>> (англ. stellar coils).}.
\vs p015 3:6 Ядро физической системы, к которой принадлежат ваше солнце и связанные с ним планеты, является центром прежней туманности Андроновер. Эта бывшая спиральная туманность была слегка искажена гравитационными нарушениями, связанными с событиями, сопровождавшими рождение вашей солнечной системы, и которые были вызваны приближением большой соседней туманности. Это --- почти столкновение --- превратило Андроновер в некое шаровидное скопление, но не уничтожило полностью двустороннюю процессию солнц и связанных с ними физических групп. Ваша солнечная система теперь занимает почти центральное положение в одном из рукавов этой деформированной спирали, примерно на полпути от центра к краю звёздного потока.
\vs p015 3:7 \pc Сектор Стрельца и все другие сектора и подразделения Орвонтона вращаются вокруг Уверсы, и некоторая путаница урантийских наблюдателей звёзд возникает из\hyp{}за иллюзий и относительных искажений, производимых следующими многочисленными вращательными движениями:
\vs p015 3:8 \li{1.}Вращение Урантии вокруг своего солнца.
\vs p015 3:9 \li{2.}Обращение вашей солнечной системы вокруг ядра бывшей туманности Андроновер.
\vs p015 3:10 \li{3.}Вращение звёздной семьи Андроновер и связанных с ней скоплений вокруг сложного ротационно\hyp{}гравитационного центра звёздного облака Небадон.
\vs p015 3:11 \li{4.}Обращение локального звёздного облака Небадон и связанных с ним творений вокруг центра их малого сектора в созвездии Стрельца.
\vs p015 3:12 \li{5.}Вращение ста малых секторов, включая малый сектор Стрельца\fnst{То есть наш малый сектор, направление на центр которого совпадает с направлением на созвездие Стрельца, если смотреть с Урантии.}, вокруг их большого сектора.
\vs p015 3:13 \li{6.}Обращение десяти больших секторов, так называемый звёздный дрейф, вокруг столичного центра Орвонтона, Уверсы.
\vs p015 3:14 \li{7.}Движение Орвонтона и шести связанных сверхвселенных вокруг Рая и Хавоны, --- совершаемое против часовой стрелки движение пространственного уровня сверхвселенной.
\vs p015 3:15 \pc Эти множественные движения бывают нескольких порядков: космические орбиты вашей планеты и вашей солнечной системы заложены изначально, --- генетически. Абсолютное движение Орвонтона против часовой стрелки также является генетическим, заложенным в архитектурных планах главной вселенной. Но промежуточные движения имеют сложное происхождение и отчасти объясняются структурной сегментацией материи\hyp{}энергии на сверхвселенные, а отчасти --- интеллектуальным и целенаправленным действием Райских организаторов сил.
\vs p015 3:16 \pc По мере приближения к Хавоне локальные вселенные сближаются друг с другом; число контуров возрастает, и наложение увеличивается слой за слоем. Но чем дальше от вечного центра, тем всё меньше и меньше систем, слоёв, контуров и вселенных.
\usection{ТУМАННОСТИ --- ПРАРОДИТЕЛИ ВСЕЛЕННЫХ}
\vs p015 4:1 
\vs p015 4:2 
\vs p015 4:3 
\vs p015 4:4 \pc 
\vs p015 4:5 \pc 
\vs p015 4:6 
\vs p015 4:7 
\vs p015 4:8 
\vs p015 4:9 
\usection{The Origin of Space Bodies}
\vs p015 5:1 
\vs p015 5:2 
\vs p015 5:3 
\vs p015 5:4 
\vs p015 5:5 
\vs p015 5:6 
\vs p015 5:7 
\vs p015 5:8 
\vs p015 5:9 
\vs p015 5:10 
\vs p015 5:11 
\vs p015 5:12 
\vs p015 5:13 
\vs p015 5:14 \pc 
\usection{The Spheres of Space}
\vs p015 6:1 
\vs p015 6:2 
\vs p015 6:3 
\vs p015 6:4 
\vs p015 6:5 
\vs p015 6:6 
\vs p015 6:7 \pc 
\vs p015 6:8 \pc 
\vs p015 6:9 
\vs p015 6:10 
\vs p015 6:11 \pc 
\vs p015 6:12 \pc 
\vs p015 6:13 
\vs p015 6:14 \pc 
\vs p015 6:15 
\vs p015 6:16 
\usection{The Architectural Spheres}
\vs p015 7:1 
\vs p015 7:2 \pc 
\vs p015 7:3 \pc 
\vs p015 7:4 
\vs p015 7:5 \pc 
\vs p015 7:6 \pc 
\vs p015 7:7 \pc 
\vs p015 7:8 \pc 
\vs p015 7:9 \pc 
\vs p015 7:10 \pc 
\vs p015 7:11 
\vs p015 7:12 \pc 
\usection{Energy Control and Regulation}
\vs p015 8:1 
\vs p015 8:2 
\vs p015 8:3 \pc 
\vs p015 8:4 
\vs p015 8:5 
\vs p015 8:6 
\vs p015 8:7 
\vs p015 8:8 
\vs p015 8:9 
\vs p015 8:10 
\usection{Circuits of the Superuniverses}
\vs p015 9:1 
\vs p015 9:2 
\vs p015 9:3 \pc 
\vs p015 9:4 
\vs p015 9:5 
\vs p015 9:6 
\vs p015 9:7 
\vs p015 9:8 
\vs p015 9:9 
\vs p015 9:10 
\vs p015 9:11 \pc 
\vs p015 9:12 
\vs p015 9:13 
\vs p015 9:14 
\vs p015 9:15 \pc 
\vs p015 9:16 
\vs p015 9:17 
\vs p015 9:18 \pc 
\usection{Rulers of the Superuniverses}
\vs p015 10:1 
\vs p015 10:2 
\vs p015 10:3 \pc 
\vs p015 10:4 
\vs p015 10:5 
\vs p015 10:6 
\vs p015 10:7 
\vs p015 10:8 
\vs p015 10:9 
\vs p015 10:10 
\vs p015 10:11 \pc 
\vs p015 10:12 
\vs p015 10:13 
\vs p015 10:14 
\vs p015 10:15 
\vs p015 10:16 
\vs p015 10:17 
\vs p015 10:18 
\vs p015 10:19 
\vs p015 10:20 
\vs p015 10:21 \pc 
\vs p015 10:22 \pc 
\vs p015 10:23 
\usection{The Deliberative Assembly}
\vs p015 11:1 
\vs p015 11:2 
\vs p015 11:3 
\usection{The Supreme Tribunals}
\vs p015 12:1 
\vs p015 12:2 
\vs p015 12:3 
\vs p015 12:4 
\usection{The Sector Governments}
\vs p015 13:1 
\vs p015 13:2 
\vs p015 13:3 
\vs p015 13:4 \pc 
\vs p015 13:5 
\vs p015 13:6 
\usection{Purposes of the Seven Superuniverses}
\vs p015 14:1 
\vs p015 14:2 
\vs p015 14:3 
\vs p015 14:4 
\vs p015 14:5 \pc 
\vs p015 14:6 
\vs p015 14:7 
\vs p015 14:8 
\vs p015 14:9 \pc 
\vsetoff
\vs p015 14:10 
\quizlink
