\upaper{43}{СОЗВЕЗДИЯ}
\uminitoc{СТОЛИЦА СОЗВЕЗДИЯ}
\uminitoc{ПРАВИТЕЛЬСТВО СОЗВЕЗДИЯ}
\uminitoc{ВСЕВЫШНИЕ НОРЛАТИАДЕКА}
\uminitoc{ГОРА СОБРАНИЯ~--- ОТ ВЕКА ВЕРНЫЙ}
\uminitoc{ОТЦЫ ЭДЕНЦИИ СО ВРЕМЕНИ ВОССТАНИЯ ЛЮЦИФЕРА}
\uminitoc{САДЫ БОГА}
\uminitoc{УНИВИТАЦИИ}
\uminitoc{УЧЕБНЫЕ МИРЫ ЭДЕНЦИИ}
\uminitoc{ГРАЖДАНСТВО НА ЭДЕНЦИИ}
\author{Малаватия Мелхиседек}
\vs p043 0:1 Урантию обычно называют 606-й Сатании в Норлатиадеке Небадона, что означает 606-й обитаемый мир в локальной системе Сатания, расположенной в созвездии Норлатиадек, одном из 100-а созвездий локальной вселенной Небадон. Так как созвездия являются первичными подразделениями локальной вселенной, их правители связывают локальные системы обитаемых миров с центральной администрацией локальной вселенной на Спасограде, а с помощью системы отражения со сверхадминистрацией От Века Древних на Уверсе.
\vs p043 0:2 \pc Правительство вашего созвездия расположено в скоплении из 771-й архитектурной сферы, центральной и крупнейшей из которых является Эденция~--- резиденция администрации Отцов Созвездия, Всевышних Норлатиадека. Сама Эденция приблизительно в 100 раз больше вашего мира. 70 больших сфер, окружающих Эденцию, примерно в 10 раз превышают размер Урантии, в то время как 10 спутников, которые вращаются вокруг каждого из этих 70-и миров, размером почти с Урантию\fnst{Обратите внимание, что эти пропорции справедливы и для архитектурных сфер локальной системы (см.\,\bibref[45:0.1]{p045 0:1}). Предполагая, что они остаются верными для скопления сфер Спасограда, мы можем вычислить общую площадь поверхности всех архитектурных миров (сравни с примечанием \bibref[37:10.1]{p037 10:1}): $10^4 + 70*10^2 + 70*6 + 100*(10^4 + 70*10^2 + 70*10) + 10\,000*(10^4 + 7*10^2 + 7*7) = 109\,277\,420$. Это более чем в 10 раз больше, общей поверхности всех эволюционных миров.}. Эти 771-а архитектурные сферы вполне сопоставимы по размеру с аналогичными сферами других созвездий.\tunemarkup{pictures}{\begin{figure}[H]\centering\includegraphics[width=0.98\columnwidth]{images/EDENTIA-700.jpg}\caption{Эденция от Гари Тонга}\end{figure}}
\vs p043 0:3 \pc На Эденции принята система исчисления времени и измерения расстояний Спасограда и, подобно сферам вселенской столицы, столичные миры созвездия полностью обеспечены всеми категориями небесных разумных существ. В целом, эти личности мало чем отличаются от тех, что описаны в связи с управлением вселенной.
\vs p043 0:4 Надзирающие серафимы, третья категория ангелов локальной вселенной, назначаются на службу в созвездия. Они располагают свои центры на столичных сферах и активно служат окружающим мирам моронтийного обучения. 70 больших сфер Норлатиадека, вместе с 700-ми малыми спутниками, населены унивитациями~--- постоянными гражданами созвездия. Все эти архитектурные миры полностью управляются различными группами местной жизни, по большей части нераскрытыми, но в число которых входят эффективные спиронги и прекрасные спорнагии. Моронтийная жизнь созвездий, являясь средней точкой в процессе моронтийной подготовки, одновременно и типична, и идеальна, как это и следовало ожидать.
\usection{СТОЛИЦА СОЗВЕЗДИЯ}
\vs p043 1:1 Эденция изобилует очаровательный нагорьями, обширными возвышенностями из физической материи, увенчанными моронтийной жизнью и преисполненными духовного великолепия, но там нет тех крутых горных хребтов, что встречаются на Урантии. Там есть десятки тысяч искрящихся озёр, и многие тысячи соединяющихся друг с другом ручьёв, но нет ни огромных океанов, ни стремительных рек. Только нагорья свободны от этих поверхностных потоков.
\vs p043 1:2 Вода Эденции и похожих архитектурных сфер ничем не отличается от воды эволюционных планет. Водные системы таких сфер бывают как поверхностными, так и грунтовыми, а влага находится в постоянной циркуляции. По этим различным водным маршрутам можно обогнуть всю Эденцию, однако основной транспортной средой является атмосфера. Духовные существа естественным образом путешествуют над поверхностью сферы, в то время как моронтийные и материальные существа для преодоления атмосферы используют материальные и полуматериальные средства передвижения.
\vs p043 1:3 Эденция и связанные с ней миры обладают настоящей атмосферой, обычной смесью трёх газов, которая характерна для таких архитектурных творений и включает два элемента атмосферы Урантии плюс моронтийный газ, пригодный для дыхания моронтийных созданий. Но хотя эта атмосфера и материальна, и моронтийна, штормов и ураганов там не бывает; не бывает также ни лета, ни зимы. Это отсутствие атмосферных возмущений и сезонных колебаний позволяет украсить всё внешнее пространство на этих специально созданных мирах.
\vs p043 1:4 Физические особенности нагорий Эденции великолепны, и их красота усиливается бесконечным богатством жизни, изобилующей на всём их протяжении. За исключением нескольких довольно изолированных строений, эти нагорья не содержат ничего рукотворного. Материальные и моронтийные украшения ограничены жилыми районами. Меньшие возвышенности~--- это места расположения особых резиденций и предстают в изысканной красоте как биологического, так и моронтийного искусства.
\vs p043 1:5 \pc На вершине седьмой гряды нагорий расположены залы воскрешения Эденции, где пробуждаются восходящие смертные вторичного модифицированного порядка восхождения. Эти палаты восстановления созданий находятся под наблюдением Мелхиседеков. Первая из принимающих сфер Эденции (подобно планете Мелхиседек возле Спасограда) также имеет специальные залы воскрешения, где восстанавливаются смертные модифицированных порядков восхождения.
\vs p043 1:6 Мелхиседеки также содержат на Эденции два специальных колледжа. Один из них, школа чрезвычайных ситуаций, занимается изучением проблем, возникших в результате восстания в Сатании. Другой, школа посвящения, предназначается для досконального изучения новых проблем, связанных с завершающим посвящением Михаила на одном из миров Норлатиадека. Этот последний колледж был учреждён почти 40\,000\fnst{В тексте 1955 года было указано <<четыре тысячи>>. Исправление во втором издании кажется оправданным на основании ссылки \bibref[119:7.2]{p119 7:2} в тексте: <<Публичное объявление о том, что Михаил выбрал Урантию в качестве театра для своего последнего посвящения, было сделано вскоре после того, как мы узнали о невыполнении обязательств Адамом и Евой. Таким образом, на протяжении более тридцати пяти тысяч лет ваш мир занимал очень видное место в советах всей вселенной. Проступок был совершён около 37\,800 лет назад, поэтому <<почти сорок тысяч>> и <<более тридцати пяти тысяч>> кажутся одинаково разумными описаниями. Комитет пришёл к выводу, что проблема по происхождению здесь идентична проблеме \bibref[41:4.4]{p041 4:4} в тексте: рассматриваемое число было написано в рукописи как цифра (40\,000, а не сорок тысяч), а ошибка была вызвана потерей нуля до того, как число было отформатировано в слова для печати.} лет назад, непосредственно после провозглашения Михаилом, что Урантия выбрана в качестве мира его завершающего посвящения.
\vs p043 1:7 \pc Стеклянное море, принимающая область Эденции, находится недалеко от административного центра и окружено столичным амфитеатром. Вокруг этой области находятся управляющие центры 70-и подразделений, относящихся к делам созвездия. Половина Эденции разделена на 70 треугольных секций, границы которых сходятся в зданиях центров соответствующих секторов. Остальная часть этой сферы представляет собой один огромный природный парк~--- сады Бога.
\vs p043 1:8 Хотя во время периодических посещений Эденции твоему взору будет открыта вся планета, б\'ольшую часть времени ты будешь проводить в административном треугольнике, номер которого соответствует номеру мира твоего текущего пребывания. Ты всегда будешь радушно принят в законодательных ассамблеях в качестве наблюдателя.
\vs p043 1:9 Моронтийная область, отведённая постоянно проживающим на Эденции восходящим смертным, находится в средней области 35-го треугольника, который примыкает к центру завершителей, расположенному в 36-м треугольнике. Главный центр унивитаций занимает огромную область в средней части 34-го треугольника, непосредственно примыкающего к месту, отведённому для проживания моронтийных граждан. Как видно из этой планировки, предусмотрено размещение как минимум 70-и основных разделов небесной жизни, причём каждая из этих 70-и треугольных областей соотносится с какой\hyp{}либо из 70-и основных сфер моронтийной подготовки.
\vs p043 1:10 Стеклянное море Эденции представляет собой один огромный круглый кристалл примерно 160\,км в окружности и около 48\,км в глубину. Этот великолепный кристалл служит в качестве принимающего поля для всех транспортных серафимов и других существ, прибывающих из-за пределов сферы; такое стеклянное море значительно облегчает посадку транспортных серафимов.
\vs p043 1:11 Подобное кристаллическое поле встречается почти на всех архитектурных мирах; и кроме своей декоративной ценности, оно служит многим целям, так как используется для представления сверхвселенской системы отражения собирающимся здесь группам и как фактор в технике преобразования энергии для изменения потоков пространства и для адаптации других поступающих потоков физической энергии.
\usection{ПРАВИТЕЛЬСТВО СОЗВЕЗДИЯ}
\vs p043 2:1 Созвездия~--- это автономные единицы локальной вселенной, каждое созвездие управляется в соответствии со своими собственными законодательными актами. Когда суды Небадона выносят решения по вселенским делам, все внутренние вопросы решаются в согласии с законами, действующими в соответствующем созвездии. Эти судебные постановления Спасограда вместе с законодательными актами созвездий исполняются администраторами локальных систем.
\vs p043 2:2 Таким образом, созвездия функционируют как законодательные или законотворческие единицы, в то время как локальные системы служат как исполнительные единицы, проводящие законы в жизнь. Правительство Спасограда является высшим судебным и координирующим органом власти.
\vs p043 2:3 \pc В то время как верховная судебная функция возложена на центральную администрацию локальной вселенной, в столице каждого созвездия существует два вспомогательных, но важных суда: совет Мелхиседеков и суд Всевышнего.
\vs p043 2:4 Все судебные проблемы сначала рассматриваются советом Мелхиседеков. Двенадцать представителей этой категории, получившие определённый необходимый опыт на эволюционных планетах и на столичных мирах систем, уполномочены рассматривать улики, анализировать заявления и выносить предварительные вердикты, которые передаются в суд Всевышнего~--- правящего Отца Созвездия. Отделение смертных этого последнего суда состоит из семи судей, каждый из которых~--- восходящий смертный. Чем выше ты восходишь во вселенной, тем больше вероятность того, что ты будешь судим себе подобными.
\vs p043 2:5 \pc Законодательный орган созвездия разделён на три группы. Законодательная программа созвездия берёт начало в нижней палате восходящих, группе, возглавляемой завершителем и состоящей из 1000 смертных представителей. Каждая система для участия в этой совещательной ассамблее выдвигает по 10 представителей. На Эденции в настоящее время этот орган укомплектован не полностью.
\vs p043 2:6 Средняя палата законодателей составлена из серафических воинств и их партнёров, других детей Материнского Духа локальной вселенной. Эта группа насчитывает 100 членов и назначается надзирающими личностями, которые возглавляют различные виды деятельности, осуществляемые этими существами в созвездии.
\vs p043 2:7 Совещательный или высший орган законодателей созвездия состоит из палаты пэров~--- палаты божественных Сынов. Этот корпус избирается Всевышними Отцами и насчитывает 10 членов. Только Сыны с особым опытом могут служить в этой верхней палате. Эта расследующая факты и экономящая время группа очень эффективно служит обоим низшим подразделениям законодательной ассамблеи.
\vs p043 2:8 Объединённый совет законодателей состоит из трёх членов от каждой из этих отдельных ветвей совещательной ассамблеи созвездия и возглавляется правящим младшим Всевышним. Эта группа утверждает окончательную форму всех законодательных актов и санкционирует их обнародование посредством трансляций. Утверждение этой верховной комиссией делает законодательные акты законом данной области; их постановления являются окончательными. Законодательные заключения Эденции составляют основной закон для всего Норлатиадека.
\usection{ВСЕВЫШНИЕ НОРЛАТИАДЕКА}
\vs p043 3:1 Правители созвездий принадлежат к сынам локальной вселенной категории Ворондадеков. Назначенные на действительную службу во вселенной в качестве правителей созвездий или иным образом, эти Сыны известны как \bibemph{Всевышние,} ибо из всех категорий Сынов Бога Локальной Вселенной они воплощают наивысшую административную мудрость, в сочетании с самой дальновидной и разумной преданностью. Их личная честность и преданность как группы никогда не подвергались сомнению; никогда в Небадоне не возникало недовольства со стороны Сынов Ворондадеков.
\vs p043 3:2 \pc В качестве Всевышних каждого из созвездий Небадона Гавриил назначает как минимум трёх Сынов Ворондадеков. Возглавляющий это трио известен как \bibemph{Отец Созвездия,} а двое его помощников~--- как \bibemph{старший Всевышний} и \bibemph{младший Всевышний}. Отец Созвездия правит в течение 10\,000 стандартных лет (приблизительно 50\,000 лет Урантии), прослужив до этого младшим помощником и старшим помощником в течение таких же периодов времени.
\vs p043 3:3 Псалмопевец знал, что Эденция управляется тремя Отцами Созвездия и соответственно говорил об их обители во множественном числе: <<Есть река, потоки которой радуют город Бога, самое святое место в жилищах Всевышних>>\fnst{В настоящем массоретском тексте Псалмов~46:5 (4 на англ.), а также во всех его древних версиях, которые я исследовал (греческая Септуагинта, латинская Вульгата, древнеармянская, сирийская Пешитта, арамейский Таргум и старославянская) <<Всевышний>> употребляется в \bibemph{единственном,} а не во \bibemph{множественном} числе.}.
\vs p043 3:4 \pc Веками на Урантии существует великая путаница в отношении различных правителей вселенной. Многие более поздние учителя принимали своих непонятных и странных племенных божеств за Всевышних Отцов. Ещё позднее евреи соединили всех этих небесных правителей в сложное Божество. Один учитель понимал, что Всевышние не были Верховными Правителями, ибо он сказал: <<Тот, кто обитает в тайном месте Всевышнего будет жить под сенью Всемогущего>>. В записях Урантии иногда очень трудно точно понять, кто упоминается под термином <<Всевышний>>. Но Даниил полностью понимал\fnst{Следующее утверждение повторяется \bibemph{три} раза в Книге Даниила: 4:14, 4:22 и 4:29.} эти вещи. Он сказал: <<Всевышний правит царством человеческим и даёт его, кому хочет>>.
\vs p043 3:5 \pc Отцы Созвездия мало занимаются индивидуумами обитаемых планет, но они тесно связаны с теми законодательными и законотворческими функциями созвездий, которые в значительной степени касаются каждой смертной \bibemph{расы} и национальной \bibemph{группы} обитаемых миров.
\vs p043 3:6 Хотя режим созвездия занимает промежуточное положение между вами и администрацией вселенной, как индивидуумы вы обычно почти не связаны с правительством созвездия. При нормальных обстоятельствах наибольший интерес для вас представляла бы локальная система, Сатания; но временно, из-за определённых условий возникших в системе и на планете в результате восстания Люцифера, Урантия тесно связана с правителями созвездия.
\vs p043 3:7 Всевышние Эденции во время отступничества Люцифера захватили контроль над определёнными фазами планетарной власти на восставших мирах. Они продолжают осуществлять эту власть, и От Века Древние уже давно утвердили такую передачу контроля над этими сбившимися с пути мирами. Они несомненно продолжат осуществлять эти взятые на себя юридические полномочия до тех пор, пока жив Люцифер. Обычно значительная часть этой власти в лояльной системе принадлежит Властелину Системы.
\vs p043 3:8 Но есть и другая причина особой связи Урантии со Всевышними. Когда Михаил, Сын Создатель, выполнял свою заключительную миссию посвящения, преемник Люцифера не обладал всей полнотой власти в локальной системе, поэтому все дела Урантии, касающиеся посвящения Михаила, находились непосредственно под наблюдением Всевышних Норлатиадека.
\usection{ГОРА СОБРАНИЯ~--- ОТ ВЕКА ВЕРНЫЙ}
\vs p043 4:1 Святейшая гора собрания~--- это место обитания От Века Верного, действующего на Эденции представителя Райской Троицы.
\vs p043 4:2 От Века Верный, Сын Райской Троицы, присутствует на Эденции как личный представитель Иммануила с момента создания этого столичного мира. Неизменно От Века Верный остаётся правой рукой Отцов Созвездия консультируя их, но не предлагает совета, если его об этом не просят. Высокие Сыны Рая никогда не участвуют в управлении делами локальной вселенной, кроме как по просьбе действующих правителей таких владений. И для Всевышних созвездия От Века Верный является тем же, чем От Века Единый~--- для Сына Создателя.
\vs p043 4:3 Резиденция От Века Верного Эденции представляет собой центр Райской системы вневселенской связи и информации данного созвездия. Эти Троичные Сыны со своим персоналом личностей Хавоны и Рая во взаимодействии с наблюдающим От Века Единым, поддерживают прямую и постоянную связь с представителями своей категории во всех вселенных, вплоть до Хавоны и Рая.
\vs p043 4:4 Святейшая гора изысканно красива и великолепно обустроена, но сама резиденция Райского Сына скромна по сравнению с центральной обителью Всевышних и окружающими её 70-ю сооружениями, составляющими жилой комплекс Сынов Ворондадеков. Эти оборудованные места исключительно жилые; они полностью отделены от обширных зданий административного центра, где решаются дела созвездия.
\vs p043 4:5 Резиденция От Века Верного на Эденции расположена к северу от этих резиденций Всевышних и известна как <<гора Райского собрания>>. На этом освящённом нагорье, периодически собираются восходящие смертные, чтобы послушать рассказ этого Сына Рая о долгом и интригующем путешествии восходящих смертных через миллиард совершенных миров Хавоны и далее~--- к неописуемым наслаждениям Рая. И именно на этих особых встречах на Горе Собрания моронтийные смертные полнее знакомятся с различными группами личностей, происходящими из центральной вселенной.
\vs p043 4:6 Вероломный Люцифер, некогда правитель Сатании, заявляя о своих притязаниях на расширение полномочий, стремился отстранить все высшие категории сыновства из плана управления локальной вселенной. Он вынашивал в своём сердце замысел\fnst{Исайя~14:13-14: <<А говорил в сердце своём: <<взойду на небо, выше звёзд Божиих вознесу престол мой и сяду на горе в сонме богов, на краю севера; взойду на высоты облачные, буду подобен Всевышнему>>. (Син. Пер.)}, говоря: <<Вознесу трон свой выше Сынов Бога; сяду на горе собрания на севере; буду подобен Всевышнему>>.
\vs p043 4:7 \pc 100 Властелинов Систем периодически собираются на конклавы Эденции, где совещаются о благоденствии созвездия. После восстания в Сатании главные мятежники Иерусема имели обыкновение появляться на этих советах Эденции, как и в прежние времена. И не находилось никакого способа прекратить эту бесцеремонную наглость вплоть до окончания посвящения Михаила на Урантии и последующего принятия им неограниченного суверенитета над всем Небадоном. Никогда с того дня эти подстрекатели ко греху не допускались в проходящие на Эденции советы лояльных Властелинов Систем.
\vs p043 4:8 То, что учителя древности знали об этих вещах, демонстрирует следующая запись\fnst{Об этом упоминается в Иов~1:6 и Иов~2:1.}: <<И был день, когда Сыны Бога пришли предстать перед Всевышними, и среди них пришёл предстать и Сатана>>. И это констатация факта, независимо от связи, вследствие которой он появляется.
\vs p043 4:9 \pc Со времени триумфа Христа весь Норлатиадек очищается от греха и мятежников. Незадолго до смерти Михаила во плоти союзник падшего Люцифера, Сатана, пытался посетить такой конклав Эденции, но твёрдость в неприятии главных мятежников достигла такой точки, когда двери сочувствия почти повсеместно закрылись, и это выбило почву из под ног супостатов Сатании. Когда не остаётся открытой двери для восприятия зла, исчезает и возможность для разгула греха. Двери сердец всей Эденции захлопнулись для Сатаны; он был единодушно отвергнут собравшимися Властелинами Систем, и это случилось именно в тот момент, когда Сын Человеческий <<увидел Сатану, упавшего как молния с небес>>.
\vs p043 4:10 После восстания Люцифера вблизи резиденции От Века Верного было возведено новое сооружение. Это временное строение служит центром Всевышнего связного, который действует в тесном контакте с Райским Сыном в качестве советника правительства созвездия по всем вопросам, касающимся курса и отношения категории От Века к греху и восстанию.
\usection{ОТЦЫ ЭДЕНЦИИ СО ВРЕМЕНИ ВОССТАНИЯ ЛЮЦИФЕРА}
\vs p043 5:1 Периодическая смена Всевышних на Эденции была приостановлена во время восстания Люцифера. Сейчас у нас те же правители, которые исполняли свои обязанности в то время. Мы предполагаем, что состав правителей будет оставаться неизменным вплоть до окончательной ликвидации Люцифера и его сообщников.
\vs p043 5:2 Однако нынешнее правительство созвездия было расширено до 12-ти Сынов категории Ворондадек. В состав 12-ти входят:
\vs p043 5:3 \li{1.}Отец Созвездия. Нынешний Всевышний правитель Норлатиадека~--- номер 617\,318-й категории Ворондадек Небадона. Прежде чем приступить к своим обязанностям на Эденции, он побывал на службе во многих созвездиях по всей нашей локальной вселенной.
\vs p043 5:4 \li{2.}Старший Всевышний помощник.
\vs p043 5:5 \li{3.}Младший Всевышний помощник.
\vs p043 5:6 \li{4.}Всевышний советник, личный представитель Михаила со времени достижения им статуса Сына Властелина.
\vs p043 5:7 \li{5.}Всевышний исполнитель, личный представитель Гавриила, постоянно находящийся на Эденции со времени восстания Люцифера.
\vs p043 5:8 \li{6.}Всевышний глава планетарных наблюдателей, руководитель наблюдателей Ворондадеков, размещённых на изолированных мирах Сатании.
\vs p043 5:9 \li{7.}Всевышний арбитр, Сын Ворондадек, на которого возложена обязанность урегулирования всех трудностей, связанных с последствиями восстания в созвездии.
\vs p043 5:10 \li{8.}Всевышний чрезвычайный администратор Сын Ворондадек, которому поручено адаптировать чрезвычайные постановления законодательной власти Норлатиадека для изолированных в результате восстания миров Сатании.
\vs p043 5:11 \li{9.}Всевышний посредник, Сын Ворондадек, назначенный для гармонизации особенностей, связанных с посвящением на Урантии, с обычным управлением созвездием. Необходимость в функционировании этого Сына возникла в связи с определённой деятельностью архангелов и многих других необычных видов служения на Урантии, а также особой деятельностью Блестящих Вечерних Звёзд на Иерусеме.
\vs p043 5:12 \li{10.}Всевышний судья\hyp{}адвокат, глава чрезвычайного суда, посвящённого урегулированию специфических проблем Норлатиадека, возникших из-за неразберихи, последовавшей за восстанием в Сатании.
\vs p043 5:13 \li{11.}Всевышний связной, Сын Ворондадек, прикреплённый к правителям Эденции, но назначенный в качестве специального советника к От Века Верному касательно наилучшего курса решения проблем, связанных с восстанием и нелояльностью созданий.
\vs p043 5:14 \li{12.}Всевышний руководитель, президент чрезвычайного совета Эденции. Все личности, назначенные в Норлатиадек из-за переворота в Сатании, составляют чрезвычайный совет, а возглавляет его Сын Ворондадек обладающий исключительным опытом.
\vs p043 5:15 И это без учёта многочисленных Ворондадеков~--- посланников созвездий Небадона, и других представителей, постоянно проживающих на Эденции.
\vs p043 5:16 \pc Со времени восстания Люцифера, Отцы Эденции всегда проявляли особую заботу об Урантии и других изолированных мирах Сатании. Давным\hyp{}давно пророк\fnst{Цитата взята из противоречивого стиха Второзакония 32:8, который на иврите заканчивается словами <<по числу сынов Израилевых>>, а в греческой Септуагинте он заканчивается словами <<по числу ангелов Божиих>>. Обратите внимание, что настоящий текст просто опускает всю эту фразу.} осознал контролирующую руку Отцов Созвездия в делах народов: <<Когда Всевышний разделил для народов их наследство, когда он разделил сынов Адама, он поставил границы для народов>>.
\vs p043 5:17 Каждый находящийся на карантине или изолированный мир имеет Сына Ворондадека, действующего в качестве наблюдателя. Он не участвует в планетарном управлении, за исключением тех случаев, когда Отец Созвездия приказывает ему вмешаться в дела наций. Действительно, именно этот Всевышний наблюдатель является тем, кто <<правит в царствах людей>>. Урантия~--- один из изолированных миров Норлатиадека, и со времени предательства Калигастии на планете постоянно находится наблюдатель Ворондадек. Когда Макивента Мелхиседек служил в полуматериальной форме на Урантии, он почтительно оказывал уважение исполнявшему в то время свои обязанности Всевышнему наблюдателю, как записано: <<И Мелхиседек, царь Салима, был священником Всевышнего>>. Мелхиседек раскрыл отношение этого Всевышнего наблюдателя к Аврааму, когда сказал: <<И благословен Всевышний, который предал врагов твоих в руки твои>>.
\usection{САДЫ БОГА}
\vs p043 6:1 Особенным украшением столиц систем служат материальные и минеральные сооружения, в то время как столичный мир вселенной больше отражает духовное великолепие, а столицы созвездий являют собой кульминацию моронтийной деятельности и живых украшений. На столичных мирах созвездий главным образом используется живое украшение, и именно это преобладание жизни~--- ботанической артистичности~--- является причиной того, что эти миры называются <<садами Бога>>.
\vs p043 6:2 \pc Около половины Эденции занимают изысканные сады Всевышних, и эти сады относятся к числу самых восхитительных моронтийных творений локальной вселенной. Это объясняет, почему необычайно красивые места обитаемых миров Норлатиадека так часто называют <<Эдемским садом>>.
\vs p043 6:3 Центральное место этого великолепного сада занимает храм поклонения Всевышних. Псалмопевец, должно быть, что-то знал об этом, ибо писал: <<Кто взойдёт на гору Всевышних? Кто будет стоять на этом святом месте? Тот, у кого чистые руки и чистое сердце, кто не возносил душу свою к суете и не клялся ложно>>\fnst{См. Псалом~23:3--4 <<Кто взойдёт на гору Господню, или кто станет на святом месте Его? Тот, у которого руки неповинны и сердце чисто, кто не клялся душою своею напрасно и не божился ложно>>. (Син. Пер.)}. Каждый десятый день отдыха вся Эденция, во главе со Всевышними, пребывает у этой святыни в благоговейном размышлении о Боге Верховном.
\vs p043 6:4 \pc Архитектурные миры населены десятью формами материальной жизни. На Урантии есть растительный и животный мир, но на таком мире как Эденция существует десять отделов материальных категорий жизни. Если бы ты мог взглянуть на эти десять типов жизни Эденции, ты бы сразу отнёс первые три к растительным и последние три~--- к животным, но совершенно не смог бы понять природу промежуточных четырёх групп столь разнообразных и очаровательных форм жизни.
\vs p043 6:5 Даже определённо животная жизнь очень отличается от соответствующей жизни эволюционных миров, отличается настолько, что совершенно невозможно описать смертному разуму уникальный характер и нежный нрав этих неговорящих созданий. Существуют тысячи и тысячи живых существ, которых ты даже не можешь вообразить. Всё творение животного мира полностью отличается от примитивных видов животных эволюционных планет. Но вся эта животная жизнь чрезвычайно разумна и исключительно услужлива и все разнообразные виды удивительно ласковы и трогательно общительны. На таких архитектурных мирах не существует хищных созданий; на всей Эденции нет ничего, что могло бы вызвать страх какого\hyp{}либо живого существа.
\vs p043 6:6 Растительная жизнь, состоящая как из материальных, так и из моронтийных разновидностей, также существенно отличается от растительной жизни Урантии. Материальные растения имеют характерную зелёную окраску, а моронтийные эквиваленты растительной жизни имеют фиолетовый или светло\hyp{}лиловый тон различных оттенков и отливов. Такая моронтийная растительность является чистым приращением энергии; съеденная, она усваивается без остатка.
\vs p043 6:7 Наделённые 10-ю типами физической жизни, не говоря уже о моронтийных разновидностях, эти архитектурные миры предоставляют огромные возможности для биологического украшения ландшафта, а также материальных и моронтийных строений. Небесные мастеровые руководят местными спорнагиями в этой всесторонней работе по ботаническому оформлению и биологическому украшению. В то время как вашим художникам для изображения своих концепций приходится прибегать к инертным краскам и безжизненному мрамору, небесные ремесленники и унивитации чаще используют живые материалы для представления своих идей и воплощения своих идеалов.
\vs p043 6:8 Если тебе нравятся цветы, кустарники и деревья Урантии, то ты будешь любоваться ботанической красотой и цветочным великолепием божественных садов Эденции. Но попытка передать смертному разуму адекватное представление об этих красотах небесных миров выше моих возможностей. Воистину, глаз не видел такого великолепия, которое ожидает тебя по прибытии на эти миры в приключении восхождения смертных.
\usection{УНИВИТАЦИИ}
\vs p043 7:1 Унивитации являются постоянными гражданами Эденции и связанных с ней миров, и все 770 миров, окружающих столицу созвездия, находятся под их надзором. Эти дети Сына Создателя и Созидательного Духа планируются на уровне существования между материальным и духовным, но они не являются моронтийными созданиями. Уроженцы каждой из 70-ти больших сфер Эденции обладают различными видимыми формами, и чтобы соответствовать восходящей шкале унивитаций, формы моронтийных смертных настраиваются при каждой смене ими места жительства, по мере их последовательного перехода с одной сферы Эденции на другую от первого мира до семидесятого.
\vs p043 7:2 Духовно унивитации одинаковы; интеллектуально они различаются, как и смертные; по форме они во многом напоминают моронтийное состояние существования и созданы для функционирования в 70-ти различных категориях личности. Каждая из этих категорий унивитаций демонстрирует десять основных разновидностей интеллектуальной деятельности, и каждый из этих различных интеллектуальных типов возглавляет специальные учебные и культурные школы прогрессирующей профессиональной или практической социализации на каком\hyp{}либо из десяти спутников, вращающихся вокруг каждого из больших миров Эденции.
\vs p043 7:3 Эти 700 малых миров представляют собой технические сферы практического обучения работе всей локальной вселенной и открыты для всех классов разумных существ. Эти образовательные школы специального мастерства и технического знания предназначены не только для восходящих смертных, хотя моронтийные студенты, безусловно, составляют наибольшую группу из всех тех, кто посещает эти курсы обучения. Когда тебя примут на любом из 70-ти больших миров социальной культуры, тебе немедленно разрешат посещение каждого из десяти окружающих спутников.
\vs p043 7:4 В различных гостящих колониях восходящие моронтийные смертные преобладают среди управляющих реверсией, но унивитации представляют самую большую группу, связанную с небадонским корпусом небесных мастеровых. Во всём Орвонтоне никакие внехавонские существа, за исключением абандонтеров Уверсы, не могут сравниться с унивитациями в художественном мастерстве, социальной способности к адаптации и одарённости в умении согласования.
\vs p043 7:5 Эти граждане созвездия фактически не являются членами корпуса мастеровых, но они широко сотрудничают со всеми группами и вносят большой вклад в превращение миров созвездия в главные сферы реализации великолепных художественных возможностей переходной культуры. Они не функционируют за пределами столичных миров созвездия.
\usection{УЧЕБНЫЕ МИРЫ ЭДЕНЦИИ}
\vs p043 8:1 
\vs p043 8:2 
\vs p043 8:3 
\vs p043 8:4 
\vs p043 8:5 
\vs p043 8:6 
\vs p043 8:7 
\vs p043 8:8 
\vs p043 8:9 
\vs p043 8:10 
\vs p043 8:11 
\vs p043 8:12 \pc 
\vs p043 8:13 
\usection{ГРАЖДАНСТВО НА ЭДЕНЦИИ}
\vs p043 9:1 
\vs p043 9:2 \pc 
\vs p043 9:3 
\vs p043 9:4 \pc 
\vs p043 9:5 
\vsetoff
\vs p043 9:6 
\quizlink
