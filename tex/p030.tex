\upaper{30}{ЛИЧНОСТИ БОЛЬШОЙ ВСЕЛЕННОЙ}
\uminitoc{РАЙСКАЯ КЛАССИФИКАЦИЯ ЖИВЫХ СУЩЕСТВ}
\uminitoc{РЕЕСТР ЛИЧНОСТЕЙ УВЕРСЫ}
\uminitoc{ГОСТЯЩИЕ КОЛОНИИ}
\uminitoc{ВОСХОДЯЩИЕ СМЕРТНЫЕ}
\author{Могущественный Посланник}
\vs p030 0:1 Личности и не являющиеся личностями сущности, функционирующие в настоящее время на Рае и в большой вселенной, составляют почти безграничное количество живых существ. Даже количество основных категорий и типов поразило бы человеческое воображение, не говоря уже о бесчисленных подтипах и разновидностях. Тем не менее желательно дать представление о двух основных классификациях живых существ: вариант Райской классификации и сокращённую форму Реестра Личностей Уверсы.
\vs p030 0:2 Невозможно сформулировать исчерпывающую и полностью последовательную классификацию личностей большой вселенной, потому что не \bibemph{все} группы раскрыты. Потребовалось бы множество дополнительных документов, чтобы охватить дальнейшее откровение, необходимое для систематической классификации всех групп. Такое концептуальное расширение вряд ли было бы желательным, поскольку оно лишило бы мыслящих смертных следующего тысячелетия того стимула к творческим размышлениям, который дают эти частично раскрытые концепции. Лучше, если у человека не будет чрезмерного откровения; это подавляет воображение.
\usection{РАЙСКАЯ КЛАССИФИКАЦИЯ ЖИВЫХ СУЩЕСТВ}
\vs p030 1:1 Живые существа классифицируются на Рае в соответствии с присущими им и достигнутыми отношениями с Райскими Божествами. Во время великих собраний центральной вселенной  и сверхвселенных присутствующие часто группируются в соответствии с происхождением: триединого происхождения или достигшие Троицы; двойственного происхождения; и одиночного происхождения. Трудно объяснить Райскую классификацию живых существ смертному разуму, но мы уполномочены представить следующее:
\vs p030 1:2 \li{I.}\bibemph{СУЩЕСТВА ТРИЕДИНОГО ПРОИСХОЖДЕНИЯ}. Существа, созданные всеми тремя Райскими Божествами~--- либо как таковые, либо в качестве Троицы,~--- а также Тринитизованный Корпус, это обозначение относится ко всем группам тринитизованных существ, раскрытых и нераскрытых.
\vs p030 1:3 \li{А.}\bibemph{Верховные Духи}.
\vs p030 1:4 \li{1.}Семь Главных Духов.
\vs p030 1:5 \li{2.}Семь Верховных Исполнителей.
\vs p030 1:6 \li{3.}Семь категорий Отражательных Духов.
\vs p030 1:7 \li{Б.}\bibemph{Стационарные Сыны Троицы}.
\vs p030 1:8 \li{1.}Тринитизованные Секреты Верховности.
\vs p030 1:9 \li{2.}От Века Вечные.
\vs p030 1:10 \li{3.}От Века Древние.
\vs p030 1:11 \li{4.}От Века Совершенные.
\vs p030 1:12 \li{5.}От Века Недавние.
\vs p030 1:13 \li{6.}От Века Единые.
\vs p030 1:14 \li{7.}От Века Верные.
\vs p030 1:15 \li{8.}Совершенствователи Мудрости.
\vs p030 1:16 \li{9.}Божественные Советники.
\vs p030 1:17 \li{10.}Всеобщие Цензоры.
\vs p030 1:18 \li{В.}\bibemph{Существа Троичного происхождения и тринитизованные существа}.
\vs p030 1:19 \li{1.}Троичные Сыны Учителя.
\vs p030 1:20 \li{2.}Вдохновлённые Троичные Духи.
\vs p030 1:21 \li{3.}Уроженцы Хавоны.
\vs p030 1:22 \li{4.}Граждане Рая.
\vs p030 1:23 \li{5.}Нераскрытые существа Троичного происхождения.
\vs p030 1:24 \li{6.}Нераскрытые существа, тринитизованные Божествами.
\vs p030 1:25 \li{7.}Тринитизованные Сыны Достижения.
\vs p030 1:26 \li{8.}Тринитизованные Сыны Отбора.
\vs p030 1:27 \li{9.}Тринитизованные Сыны Совершенства.
\vs p030 1:28 \li{10.}Сыны, тринитизованные созданиями.
\vs p030 1:29 \li{II.}\bibemph{СУЩЕСТВА ДВОЙСТВЕННОГО ПРОИСХОЖДЕНИЯ}. Существа, происходящие от любых двух Райских Божеств или иным образом созданные любыми двумя существами прямого или косвенного происхождения от Райских Божеств.
\vs p030 1:30 \li{А.}\bibemph{Нисходящие категории}.
\vs p030 1:31 \li{1.}Сыны Создатели.
\vs p030 1:32 \li{2.}Сыны Повелители.
\vs p030 1:33 \li{3.}Яркие Утренние Звёзды.
\vs p030 1:34 \li{4.}Отцы Мелхиседеки.
\vs p030 1:35 \li{5.}Мелхиседеки.
\vs p030 1:36 \li{6.}Ворондадеки.
\vs p030 1:37 \li{7.}Ланонандеки.
\vs p030 1:38 \li{8.}Блистательные Вечерние Звёзды.
\vs p030 1:39 \li{9.}Архангелы.
\vs p030 1:40 \li{10.}Носители Жизни.
\vs p030 1:41 \li{11.}Нераскрытые Вселенские Помощники.
\vs p030 1:42 \li{12.}Нераскрытые Сыны Бога.
\vs p030 1:43 \li{Б.}\bibemph{Стационарные категории}.
\vs p030 1:44 \li{1.}Абандонтеры.
\vs p030 1:45 \li{2.}Сусации.
\vs p030 1:46 \li{3.}Унивитации.
\vs p030 1:47 \li{4.}Спиронги.
\vs p030 1:48 \li{5.}Нераскрытые существа двойственного происхождения.
\vs p030 1:49 \li{В.}\bibemph{Восходящие категории}.
\vs p030 1:50 \li{1.}Смертные, слившиеся с Настройщиком.
\vs p030 1:51 \li{2.}Смертные, слившиеся с Сыном.
\vs p030 1:52 \li{3.}Смертные, слившиеся с Духом.
\vs p030 1:53 \li{4.}Преображённые промежуточные создания.
\vs p030 1:54 \li{5.}Нераскрытые восходящие существа.
\vs p030 1:55 \li{III.}\bibemph{СУЩЕСТВА ОДИНОЧНОГО ПРОИСХОЖДЕНИЯ}. Существа, происходящие от одного из Райских Божеств или иным образом созданные каким\hyp{}либо одним существом прямого или косвенного происхождения от Райских Божеств.
\vs p030 1:56 \li{А.}\bibemph{Верховные Духи}.
\vs p030 1:57 \li{1.}Гравитационные Посланники.
\vs p030 1:58 \li{2.}Семь Духов Контуров Хавоны.
\vs p030 1:59 \li{3.}Двенадцатичастные Помощники Контуров Хавоны.
\vs p030 1:60 \li{4.}Помощники Отражательного Изображения.
\vs p030 1:61 \li{5.}Вселенские Материнские Духи.
\vs p030 1:62 \li{6.}Семичастные Адъютанты Разумо\hyp{}Духи.
\vs p030 1:63 \li{7.}Нераскрытые существа, происходящие от Божеств.
\vs p030 1:64 \li{Б.}\bibemph{Восходящие категории}.
\vs p030 1:65 \li{1.}Персонализированные Настройщики.
\vs p030 1:66 \li{2.}Восходящие Материальные сыны.
\vs p030 1:67 \li{3.}Эволюционные серафимы.
\vs p030 1:68 \li{4.}Эволюционные херувимы.
\vs p030 1:69 \li{5.}Нераскрытые восходящие создания.
\vs p030 1:70 \li{В.}\bibemph{Семейство Бесконечного Духа}.
\vs p030 1:71 \li{1.}Одинокие Посланники.
\vs p030 1:72 \li{2.}Смотрители Вселенских Контуров.
\vs p030 1:73 \li{3.}Управляющие Переписью.
\vs p030 1:74 \li{4.}Личные Помощники Бесконечного Духа.
\vs p030 1:75 \li{5.}Младшие Инспекторы.
\vs p030 1:76 \li{6.}Назначенные Стражи.
\vs p030 1:77 \li{7.}Проводники Выпускников.
\vs p030 1:78 \li{8.}Сервиталы Хавоны.
\vs p030 1:79 \li{9.}Всеобщие Миротворцы.
\vs p030 1:80 \li{10.}Моронтийные Спутники.
\vs p030 1:81 \li{11.}Супернафимы.
\vs p030 1:82 \li{12.}Секонафимы.
\vs p030 1:83 \li{13.}Терциафимы.
\vs p030 1:84 \li{14.}Омниафимы.
\vs p030 1:85 \li{15.}Серафимы.
\vs p030 1:86 \li{16.}Херувимы и сановимы.
\vs p030 1:87 \li{17.}Нераскрытые существа, происходящие от Духа.
\vs p030 1:88 \li{18.}Семь Верховных Управляющих Мощью.
\vs p030 1:89 \li{19.}Верховные Центры Мощи.
\vs p030 1:90 \li{20.}Главные Физические Регуляторы.
\vs p030 1:91 \li{21.}Управляющие Моронтийной Мощью.
\vs p030 1:92 \li{IV.}\bibemph{ВОЗНИКШИЕ ТРАНСЦЕНДЕНТАЛЬНЫЕ СУЩЕСТВА}. На Рае можно найти великое множество трансцендентальных существ, происхождение которых обычно не раскрывается вселенным времени и пространства, пока они не утверждаются в свете и жизни. Эти Трансценденталы не являются ни создателями, ни созданиями; они~--- \bibemph{возникшие} [\bibemph{eventuated}] дети божественности, предельности и вечности. Эти <<возникшие>> [``eventuators''] не являются ни конечными, ни бесконечными~--- они \bibemph{абсонитны;} а абсонитность~--- это ни бесконечность, ни абсолютность.
\vs p030 1:93 Эти несозданные несоздатели всегда верны Райской Троице и послушны Предельному. Они существуют на четырёх предельных уровнях личностной активности и функционируют на семи уровнях абсонитного в двенадцати больших подразделениях, состоящих из тысячи основных рабочих групп по семь классов в каждой. Эти возникшие существа включают в себя следующие категории:
\vs p030 1:94 \li{1.}Зодчие Главной Вселенной.
\vs p030 1:95 \li{2.}Трансцендентальные Писцы.
\vs p030 1:96 \li{3.}Прочие Трансценденталы.
\vs p030 1:97 \li{4.}Первичные Возникшие Главные Организаторы Силы.
\vs p030 1:98 \li{5.}Младшие Трансцендентальные Главные Организаторы Силы.
\vs p030 1:99 \pc Бог, как сверхличность, возникает; Бог, как личность, творит; Бог, как доличность, фрагментируется; и такой фрагмент его самого, как Настройщик, развивает душу духа на базе материального и смертного разума в соответствии с добровольным выбором личности, которая была дарована такому смертному созданию родительским актом Бога как Отца.
\vs p030 1:100 \li{V.}\bibemph{ФРАГМЕНТИРОВАННЫЕ СУЩНОСТИ БОЖЕСТВА}. Эта категория живого существования, происходящая от Всеобщего Отца, лучше всего представлена Настройщиками Мыслей, хотя эти сущности ни в коем случае не являются единственными фрагментами доличностной реальности Первого Источника и Центра. Функции фрагментов, не являющихся Настройщиками, многочисленны и мало известны. Слияние с Настройщиком или другим подобным фрагментом составляет \bibemph{существо, слившееся с Отцом}.
\vs p030 1:101 Здесь следует отметить фрагментации предр\'азумного духа Третьего Источника и Центра, хотя они едва ли сравнимы с фрагментами Отца. Такие сущности очень сильно отличаются от Настройщиков; как таковые, они не обитают на Духограде и не перемещаются по контурам гравитации разума; не вселяются они и в смертные создания в течение жизни во плоти. Они не являются доличностными в том смысле, в каком являются Настройщики, но такие фрагменты предр\'азумного духа даруются некоторым выжившим смертным, и слияние с ними делает их \bibemph{слившимися с Духом смертными,}~--- в отличие от смертных, слившихся с Настройщиком.
\vs p030 1:102 Ещё труднее описать индивидуализированный дух Сына Создателя, союз с которым делает создание \bibemph{смертным, слившимся с Сыном}. Существуют и другие фрагментации Божества.
\vs p030 1:103 \li{VI.}\bibemph{СВЕРХЛИЧНОСТНЫЕ СУЩЕСТВА}. Во вселенной вселенных существует великое воинство неличностных существ божественного происхождения и разнообразного служения. Некоторые из этих существ обитают на Райских мирах Сына; другие, как сверхличностные представители Вечного Сына, встречаются повсюду. Они по большей части не упоминаются в этих повествованиях, и было бы совершенно бесполезно пытаться описать их \bibemph{личностным} существам.
\vs p030 1:104 \li{VII.}\bibemph{НЕКЛАССИФИЦИРОВАННЫЕ И НЕРАСКРЫТЫЕ КАТЕГОРИИ}. В течение нынешней вселенской эпохи было бы невозможно уложить всех существ, личностных и иных, в рамки классификации, относящейся к нынешней вселенской эпохе; кроме того, не все такие категории раскрыты в этих повествованиях; поэтому многочисленные категории были опущены в этих списках. Обрати внимание на следующие:
\vs p030 1:105 Завершитель Вселенского Предназначения.
\vs p030 1:106 Условные Наместники Предельного.
\vs p030 1:107 Безусловные Смотрители Верховного.
\vs p030 1:108 Нераскрытые Созидательные Силы От Века Древних.
\vs p030 1:109 Мажестон Рая.
\vs p030 1:110 Неназванные Отражательные Связные Мажестона.
\vs p030 1:111 Мидсонитные категории локальных вселенных.
\vs p030 1:112 \pc Не нужно придавать особого значения перечислению этих категорий вместе, кроме того, что ни один из них не фигурирует в Райской классификации в том виде, в котором она здесь раскрыта. Это лишь немногие из не вошедших в классификацию; тебе ещё предстоит узнать о многих нераскрытых.
\vs p030 1:113 Существуют различные духи: духи\hyp{}сущности, духи\hyp{}присутствия, личностные духи, доличностные духи, сверхличностные духи, духи\hyp{}существования, духи\hyp{}личности~--- но здесь не адекватны\fnst{То, что язык и интеллект смертных не адекватны очевидно хотя бы из того, что совершенно неясно, почему <<личностные духи>> и <<духи\hyp{}личности>> упомянуты отдельно.} ни язык смертных, ни смертный интеллект. Тем не менее мы можем отметить, что не существует личностей <<чистого разума>>; ни одна сущность не обладает личностью, если она не была наделена ею Богом, который есть дух. Любая разумная сущность, не связанная ни с духовной, ни с физической энергией, не является личностью. Но в том же смысле, в каком существуют духи\hyp{}личности, которые имеют разум, существуют разумы\hyp{}личности, которые имеют дух. Мажестон и его партнёры являются довольно хорошими иллюстрациями существ, у которых разум доминирует, но есть и лучшие иллюстрации этого типа личности, которые вам неизвестны. Существуют даже целые нераскрытые категории таких \bibemph{личностей разума,} но они всегда связаны с духом. Некоторые другие нераскрытые создания можно было бы назвать \bibemph{личностями разумо\hyp{}энергетическими и физико\hyp{}энергетическими}. Этот тип существ невосприимчив к гравитации духа, но тем не менее является истинной личностью~--- находится в контуре Отца.
\vs p030 1:114 \pc Эти документы не могут даже начать исчерпывающее изложение рассказа о живых существах, создателях, возникших и существующих иным образом существ, которые живут, поклоняются и служат в кишащих созданиями вселенных времени и в центральной вселенной вечности. Вы, смертные, являетесь личностями; поэтому мы можем описывать существа, которые \bibemph{персонализированы,} но как можно объяснить вам \bibemph{абсонитизированное} существо?
\usection{РЕЕСТР ЛИЧНОСТЕЙ УВЕРСЫ}
\vs p030 2:1 Божественная семья живых существ зарегистрирована на Уверсе в семи основных разделах:
\vs p030 2:2 \li{1.}Райские Божества.
\vs p030 2:3 \li{2.}Верховные Духи.
\vs p030 2:4 \li{3.}Существа Троичного происхождения.
\vs p030 2:5 \li{4.}Сыны Бога.
\vs p030 2:6 \li{5.}Личности Бесконечного Духа.
\vs p030 2:7 \li{6.}Управляющие Вселенской Мощью.
\vs p030 2:8 \li{7.}Корпус Постоянного Гражданства.
\vs p030 2:9 \pc Эти группы волевых существ делятся на многочисленные классы и малые подразделения. Однако изложение данной классификации личностей большой вселенной главным образом связано с представлением тех категорий разумных существ, которые были раскрыты в этих повествованиях, большинство из которых будут встречаться в опыте восхождения смертных времени на их постепенном подъёме к Раю. В нижеследующих списках не упоминаются обширные категории вселенских существ, которые выполняют свою работу вне плана восхождения смертных.
\vs p030 2:10 \li{I.}\bibemph{РАЙСКИЕ БОЖЕСТВА}.
\vs p030 2:11 \li{1.}Всеобщий Отец.
\vs p030 2:12 \li{2.}Вечный Сын.
\vs p030 2:13 \li{3.}Бесконечный Дух.
\vs p030 2:14 \li{II.}\bibemph{ВЕРХОВНЫЕ ДУХИ}.
\vs p030 2:15 \li{1.}Семь Главных Духов.
\vs p030 2:16 \li{2.}Семь Верховных Исполнителей.
\vs p030 2:17 \li{3.}Семь групп Отражательных Духов.
\vs p030 2:18 \li{4.}Помощники Отражательного Изображения.
\vs p030 2:19 \li{5.}Семь Духов Контуров.
\vs p030 2:20 \li{6.}Созидательные Духи локальных вселенных.
\vs p030 2:21 \li{7.}Адъютанты Разумо\hyp{}Духи.
\vs p030 2:22 \tunemarkup{pgnexus10}{\kern-4pt}\li{III.}\tunemarkup{pgnexus10}{\kern-5pt}\bibemph{СУЩЕСТВА ТРОИЧНОГО ПРОИСХОЖДЕНИЯ}.
\vs p030 2:23 \li{1.}Тринитизованные Секреты Верховности.
\vs p030 2:24 \li{2.}От Века Вечные.
\vs p030 2:25 \li{3.}От Века Древние.
\vs p030 2:26 \li{4.}От Века Совершенные.
\vs p030 2:27 \li{5.}От Века Недавние.
\vs p030 2:28 \li{6.}От Века Единые.
\vs p030 2:29 \li{7.}От Века Верные.
\vs p030 2:30 \li{8.}Троичные Сыны Учителя.
\vs p030 2:31 \li{9.}Совершенствователи Мудрости.
\vs p030 2:32 \li{10.}Божественные Советники.
\vs p030 2:33 \li{11.}Вселенские Цензоры.
\vs p030 2:34 \li{12.}Вдохновлённые Троичные Духи.
\vs p030 2:35 \li{13.}Уроженцы Хавоны.
\vs p030 2:36 \li{14.}Граждане Рая.
\vs p030 2:37 \li{IV.}\bibemph{СЫНЫ БОГА}.
\vs p030 2:38 \li{A.}\bibemph{Нисходящие Сыны}.
\vs p030 2:39 \li{1.}Сыны Создатели~--- Михаилы.
\vs p030 2:40 \li{2.}Сыны Повелители~--- Авоналы.
\vs p030 2:41 \li{3.}Троичные Сыны Учителя~--- Дайналы.
\vs p030 2:42 \li{4.}Сыны Мелхиседеки.
\vs p030 2:43 \li{5.}Сыны Ворондадеки.
\vs p030 2:44 \li{6.}Сыны Ланонандеки.
\vs p030 2:45 \li{7.}Сыны Носители Жизни.
\vs p030 2:46 \li{Б.}\bibemph{Восходящие Сыны}.
\vs p030 2:47 \li{1.}Слившиеся с Отцом смертные.
\vs p030 2:48 \li{2.}Слившиеся с Сыном смертные.
\vs p030 2:49 \li{3.}Слившиеся с Духом смертные.
\vs p030 2:50 \li{4.}Эволюционные серафимы.
\vs p030 2:51 \li{5.}Восходящие Материальные Сыны.
\vs p030 2:52 \li{6.}Преображённые промежуточные создания.
\vs p030 2:53 \li{7.}Персонализированные Настройщики.
\vs p030 2:54 \li{В.}\bibemph{Тринитизованные Сыны}.
\vs p030 2:55 \li{1.}Могущественные Посланники.
\vs p030 2:56 \li{2.}Высокоуполномоченные.
\vs p030 2:57 \li{3.}Не Имеющие Имени и Номера.
\vs p030 2:58 \li{4.}Тринитизованные Хранители.
\vs p030 2:59 \li{5.}Тринитизованные Послы.
\vs p030 2:60 \li{6.}Небесные Стражи.
\vs p030 2:61 \li{7.}Помощники Высоких Сынов.
\vs p030 2:62 \li{8.}Сыны, тринитизованные восходящими созданиями.
\vs p030 2:63 \li{9.}Тринитизованные Сыны Рая Хавоны.
\vs p030 2:64 \li{10.}Тринитизованные Сыны Предназначения.
\vs p030 2:65 \li{V.}\bibemph{ЛИЧНОСТИ БЕСКОНЕЧНОГО ДУХА}.
\vs p030 2:66 \li{А.}\bibemph{Высшие Личности Бесконечного Духа}.
\vs p030 2:67 \li{1.}Одинокие Посланники.
\vs p030 2:68 \li{2.}Смотрители Вселенских Контуров.
\vs p030 2:69 \li{3.}Управляющие Переписью.
\vs p030 2:70 \li{4.}Личные Помощники Бесконечного Духа.
\vs p030 2:71 \li{5.}Младшие Инспекторы.
\vs p030 2:72 \li{6.}Назначенные Стражи.
\vs p030 2:73 \li{7.}Проводники Выпускников.
\vs p030 2:74 \li{Б.}\bibemph{Воинства Посланников Пространства}.
\vs p030 2:75 \li{1.}Сервиталы Хавоны.
\vs p030 2:76 \li{2.}Всеобщие Миротворцы.
\vs p030 2:77 \li{3.}Технические Консультанты.
\vs p030 2:78 \li{4.}Хранители Райских Записей.
\vs p030 2:79 \li{5.}Небесные Писцы.
\vs p030 2:80 \li{6.}Моронтийные Спутники.
\vs p030 2:81 \li{7.}Райские Спутники.
\vs p030 2:82 \li{В.}\bibemph{Духи\hyp{}Помощники}.
\vs p030 2:83 \li{1.}Супернафимы.
\vs p030 2:84 \li{2.}Секонафимы.
\vs p030 2:85 \li{3.}Терциафимы.
\vs p030 2:86 \li{4.}Омниафимы.
\vs p030 2:87 \li{5.}Серафимы.
\vs p030 2:88 \li{6.}Херувимы и сановимы.
\vs p030 2:89 \li{7.}Промежуточные создания.
\vs p030 2:90 \li{VI.}\bibemph{УПРАВЛЯЮЩИЕ ВСЕЛЕНСКОЙ МОЩЬЮ}.
\vs p030 2:91 \li{А.}\bibemph{Семь Верховных Управляющих Мощью}.
\vs p030 2:92 \li{Б.}\bibemph{Верховные Центры Мощи}.
\vs p030 2:93 \li{1.}Верховные Управляющие Центрами.
\vs p030 2:94 \li{2.}Центры Хавоны.
\vs p030 2:95 \li{3.}Центры сверхвселенных.
\vs p030 2:96 \li{4.}Центры локальных вселенных.
\vs p030 2:97 \li{5.}Центры созвездий.
\vs p030 2:98 \li{6.}Центры систем.
\vs p030 2:99 \li{7.}Неклассифицированные Центры.
\vs p030 2:100 \li{В.}\bibemph{Главные Физические Регуляторы}.
\vs p030 2:101 \li{1.}Младшие Управляющие Мощью.
\vs p030 2:102 \li{2.}Механические регуляторы.
\vs p030 2:103 \li{3.}Преобразователи энергии.
\vs p030 2:104 \li{4.}Передатчики энергии.
\vs p030 2:105 \li{5.}Первичные ассоциаторы.
\vs p030 2:106 \li{6.}Вторичные диссоциаторы.
\vs p030 2:107 \li{7.}Франдаланки и хронолдеки.
\vs p030 2:108 \li{Г.}\bibemph{Управляющие Моронтийной Мощью}.
\vs p030 2:109 \li{1.}Регуляторы контуров.
\vs p030 2:110 \li{2.}Координаторы систем.
\vs p030 2:111 \li{3.}Планетарные Хранители.
\vs p030 2:112 \li{4.}Объединённые регуляторы.
\vs p030 2:113 \li{5.}Стабилизаторы связи.
\vs p030 2:114 \li{6.}Селективные сортировщики.
\vs p030 2:115 \li{7.}Младшие Регистраторы.
\vs p030 2:116 \tunemarkup{pgnexus10}{\kern4pt}\li{VII.}\bibemph{КОРПУС ПОСТОЯННОГО ГРАЖДАНСТВА}.
\vs p030 2:117 \li{1.}Планетарные промежуточные создания.
\vs p030 2:118 \li{2.}Адамические Сыны систем.
\vs p030 2:119 \li{3.}Унивитации созвездий.
\vs p030 2:120 \li{4.}Сусации локальных вселенных.
\vs p030 2:121 \li{5.}Слившиеся с Духом смертные локальных вселенных.
\vs p030 2:122 \li{6.}Абандонтеры сверхвселенных.
\vs p030 2:123 \li{7.}Слившиеся с Сыном смертные сверхвселенных.
\vs p030 2:124 \li{8.}Уроженцы Хавоны.
\vs p030 2:125 \li{9.}Уроженцы Райских сфер Духа.
\vs p030 2:126 \li{10.}Уроженцы Райских сфер Отца.
\vs p030 2:127 \li{11.}Сотворённые Граждане Рая.
\vs p030 2:128 \li{12.}Слившиеся с Настройщиком смертные Граждане Рая.
\vs p030 2:129 \pc Такова рабочая классификация личностей вселенных в том виде, как они зарегистрированы на столичном мире~--- на Уверсе.
\vs p030 2:130 \pc \bibemph{СМЕШАННЫЕ ГРУППЫ ЛИЧНОСТЕЙ}. На Уверсе существуют записи о многочисленных дополнительных группах разумных существ, существ, которые также тесно связаны с организацией и управлением большой вселенной. В число таких категорий входят следующие три смешанные группы личностей:
\vs p030 2:131 \li{А.}\bibemph{Райские Корпусы Завершения}.
\vs p030 2:132 \li{1.}Корпус Смертных Завершителей.
\vs p030 2:133 \li{2.}Корпус Райских Завершителей.
\vs p030 2:134 \li{3.}Корпус Тринитизованных Завершителей.
\vs p030 2:135 \li{4.}Корпус Совместных Тринитизованных Завершителей.
\vs p030 2:136 \li{5.}Корпус Завершителей Хавоны.
\vs p030 2:137 \li{6.}Корпус Трансцендентальных Завершителей.
\vs p030 2:138 \li{7.}Корпус Нераскрытых Сынов Предназначения.
\vs p030 2:139 \pc Смертный Корпус Завершения рассматривается в следующем и заключительном документе этой серии.
\vs p030 2:140 \li{Б.}\bibemph{Вселенские Помощники}.
\vs p030 2:141 \li{1.}Яркие Утренние Звёзды.
\vs p030 2:142 \li{2.}Блистательные Вечерние Звёзды.
\vs p030 2:143 \li{3.}Архангелы.
\vs p030 2:144 \li{4.}Всевышние Помощники.
\vs p030 2:145 \li{5.}Высокие Комиссары.
\vs p030 2:146 \li{6.}Небесные Наблюдатели.
\vs p030 2:147 \li{7.}Учителя Обительских Миров.
\vs p030 2:148 \pc На всех столичных мирах как локальных вселенных, так и сверхвселенных обеспечены условия для этих существ, выполняющих конкретные миссии для Сынов Создателей, правителей локальной вселенной. Мы радушно принимаем этих \bibemph{Вселенских Помощников} на Уверсе, но они не входят в нашу юрисдикцию. Эти эмиссары выполняют свою работу и проводят свои наблюдения под руководством Сынов Создателей. Их деятельность более полно описана в рассказе о вашей локальной вселенной.
\vs p030 2:149 \li{В.}\bibemph{Семь гостящих колоний}.
\vs p030 2:150 \li{1.}Исследователи звёзд.
\vs p030 2:151 \li{2.}Небесные мастеровые.
\vs p030 2:152 \li{3.}Управляющие реверсией.
\vs p030 2:153 \li{4.}Преподаватели подготовительных школ.
\vs p030 2:154 \li{5.}Различные резервные корпусы.
\vs p030 2:155 \li{6.}Приезжие студенты.
\vs p030 2:156 \li{7.}Восходящие пилигримы.
\vs p030 2:157 \pc Такова организация и управление этих семи групп существ на всех столичных мирах~--- от локальных систем до столиц сверхвселенных, особенно последних. Столицы семи сверхвселенных~--- это места встреч почти всех классов и категорий разумных существ. За исключением многочисленных групп жителей Рая\hyp{}Хавоны, здесь можно наблюдать и изучать волевые создания любой фазы существования.
\usection{ГОСТЯЩИЕ КОЛОНИИ}
\vs p030 3:1 Семь гостящих колоний временно проживают на архитектурных сферах в течение более или менее продолжительного времени, занятые осуществлением своих миссий и исполнением своих особых заданий. Их работа может быть описана следующим образом:
\vs p030 3:2 \li{1.}\bibemph{Исследователи звёзд}~--- небесные астрономы~--- предпочитают работать на сферах, подобных Уверсе, потому что такие специально сконструированные миры особенно благоприятны для их наблюдений и расчётов. Уверса удачно расположена для работы этой колонии не только из\hyp{}за её центрального местонахождения, но и потому, что поблизости нет гигантских живых или мёртвых солнц, которые приводили бы к возмущениям токов энергии. Эти исследователи никоим образом не связаны органично с делами сверхвселенной; они просто гости.
\vs p030 3:3 Астрономическая колония Уверсы включает индивидуумов из многих близлежащих миров, из центральной вселенной и даже из Норлатиадека. Любое существо из любого мира любой системы любой вселенной может стать исследователем звёзд, может стремиться вступить в один из корпусов небесных астрономов. Единственные необходимые условия: продолжение жизни и достаточное знание миров пространства, особенно их физических законов эволюции и контроля. Исследователи звёзд не обязаны служить в этом корпусе вечно, но никто, принятый в эту группу, не может уйти раньше одного тысячелетия по времени  Уверсы.
\vs p030 3:4 Колония звёздных наблюдателей Уверсы сейчас насчитывает более миллиона существ. Эти астрономы приходят и уходят, хотя некоторые остаются на сравнительно долгое время. Они выполняют свою работу с помощью множества механических инструментов и физических приспособлений; им также очень помогают Одинокие Посланники и другие духи\hyp{}исследователи. Эти небесные астрономы постоянно используют живых преобразователей и передатчиков энергии, а также отражательных личностей в своей работе по изучению звёзд и обследованию пространства. Они изучают все формы и фазы пространственных материальных и энергетических проявлений, и их так же сильно интересуют силовые функции, как и звёздные явления; ничто во всём пространстве не ускользает от их пристального внимания.
\vs p030 3:5 Подобные колонии астрономов можно встретить на столичных мирах секторов сверхвселенной, а также на архитектурных столицах локальных вселенных и их административных подразделениях. За исключением Рая, знание не является врождённым; понимание физической вселенной во многом зависит от наблюдений и исследований.
\vs p030 3:6 \li{2.}\bibemph{Небесные мастеровые} служат по всем семи сверхвселенным. Восходящие смертные впервые вступают в контакт с этими группами на моронтийном пути локальной вселенной, в связи с которой эти мастеровые и будут обсуждаться более подробно.
\vs p030 3:7 \li{3.}\bibemph{Управляющие реверсией} содействуют отдыху и поднятию настроения~--- возвращению к воспоминаниям о прошлом. Они очень полезны в практическом осуществлении плана восхождения смертных, особенно на ранних этапах моронтийного перехода и опыта духа. Рассказ о них относится к повествованию о пути смертных в локальной вселенной.
\vs p030 3:8 \li{4.}\bibemph{Преподаватели подготовительных школ}. Более высокий, следующий на пути восхождения мир обитания всегда поддерживает мощный корпус учителей в более низком, непосредственно предшествующем ему~--- своего рода подготовительную школу для развивающихся жителей этой сферы; это этап схемы восхождения для продвижения пилигримов времени. Эти школы, их методы обучения и проведения экзаменов совершенно не похожи на всё, что вы пытаетесь провести на Урантии.
\vs p030 3:9 Весь план восхождения по пути развития смертных характеризуется практикой передачи другим существам новой истины и опыта сразу же после их приобретения. Вы прокладываете себе путь через долгую школу Райского достижения, служа учителями ученикам, которые находятся сразу же за вами по уровню развития.
\vs p030 3:10 \li{5.}\bibemph{Различные резервные корпусы}. Огромные резервы существ, не находящихся под нашим непосредственным руководством, мобилизованы на Уверсе в качестве колонии резервных корпусов. На Уверсе действует 70 первичных подразделений этой колонии, и для получения разностороннего образования разрешается провести некоторое время с этими необыкновенными личностями. Подобные общие резервы поддерживаются на Спасограде и других вселенских столицах; они направляются на активную службу по запросу директоров соответствующих групп.
\vs p030 3:11 \li{6.}\bibemph{Приезжие студенты}. Со всей вселенной через различные столичные миры течёт непрерывный поток небесных посетителей. Индивидуально и в классах эти различные типы существ стекаются к нам в качестве наблюдателей, учеников по обмену и помощников исследователей. На Уверсе в настоящее время в этой гостевой колонии проживает более миллиарда лиц. Некоторые из этих посетителей могут задержаться на день, другие~--- на год, всё зависит от характера их миссии. В этой колонии обитают почти все классы вселенских существ, кроме личностей Создателей и моронтийных смертных.
\vs p030 3:12 Моронтийные смертные могут быть приезжими студентами только в пределах локальной вселенной своего происхождения. Они могут гостить в сверхвселенском качестве только после того, как достигнут статуса духа. Более половины посетителей в нашей колонии составляют <<транзитные>>~--- существа, путешествующие куда\hyp{}то в другое место, которые делают остановку, чтобы посетить столицу Орвонтона. Эти личности могут выполнять вселенское задание или наслаждаться периодом досуга~--- свободы от задания. Привилегия путешествий и наблюдений внутривселенского уровня~--- это часть пути всех восходящих существ. Человеческое желание путешествовать и наблюдать за новыми народами и мирами будет полностью удовлетворено во время долгого и насыщенного событиями восхождения в Рай через локальную, сверх\hyp{} и центральную вселенные.
\vs p030 3:13 \li{7.}\bibemph{Восходящие пилигримы}. Когда восходящие пилигримы назначаются на различные службы в связи с их Райским восхождением, они размещаются в качестве гостевой колонии на различных столичных сферах. Действуя повсюду в сверхвселенной, такие группы в основном самоуправляющиеся. Они представляют собой постоянно изменяющуюся колонию, охватывающую все категории эволюционных смертных и их товарищей по восхождению.
\usection{ВОСХОДЯЩИЕ СМЕРТНЫЕ}
\vs p030 4:1 Хотя выжившие смертные времени и пространства, принятые к постепенному восхождению к Раю, называются \bibemph{восходящими пилигримами,} эти эволюционные создания занимают такое важное место в данных повествованиях, что мы желаем дать здесь краткий обзор следующих семи стадий вселенского пути восхождения:
\vs p030 4:2 \li{1.}Планетарные смертные.
\vs p030 4:3 \li{2.}Выжившие спящие.
\vs p030 4:4 \li{3.}Студенты обительских миров.
\vs p030 4:5 \li{4.}Моронтийные прогрессоры.
\vs p030 4:6 \li{5.}Подопечные сверхвселенных.
\vs p030 4:7 \li{6.}Пилигримы Хавоны.
\vs p030 4:8 \li{7.}Прибывшие в Рай.
\vs p030 4:9 \pc В следующем повествовании представлен вселенский путь смертного, в котором обитает Настройщик. Слившиеся с Сыном или Духом смертные разделяют часть этого пути, но мы решили рассказать эту историю в той форме, в какой она относится к слившимся с Настройщиком смертным, поскольку такое предназначение могут ожидать все человеческие расы Урантии.
\vs p030 4:10 \li{1.}\bibemph{Планетарные смертные}. Все смертные~--- эволюционные существа животного происхождения с потенциалом восхождения. По происхождению, природе и предназначению эти различные группы и типы человеческих существ не слишком отличаются от народов Урантии. Человеческие расы каждого мира пользуются одинаковым служением Сынов Бога и присутствием духов\hyp{}помощников времени. После естественной смерти все типы восходящих созданий объединяются в одну моронтийную семью на обительских мирах.
\vs p030 4:11 \li{2.}\bibemph{Выжившие спящие}. Все смертные со статусом выживания, находящиеся под опекой личных хранителей предназначения, проходят через врата естественной смерти и, на третий период, персонализируются на обительских мирах. Те допущенные существа, которые по какой\hyp{}либо причине не смогли достичь того уровня владения разумом и наделения духовностью, который давал бы им право на личных хранителей, не могут таким образом немедленно и непосредственно отправиться в обительские миры. Такие выжившие души должны отдыхать в бессознательном сне до судного дня новой эпохи, новой диспенсации, прихода Сына Бога, чтобы сделать эпохальную перекличку и судить мир, и это общая практика по всему Небадону. О Христе Михаиле было сказано, что, когда он взошёл на небеса по завершении своей земной работы, <<Он увёл великое множество пленников>>. И эти пленники были спящими выжившими со времён Адама до дня воскресения Учителя на Урантии.
\vs p030 4:12 Течение времени не имеет значения для спящих смертных; они совершенно не осозна\'ют и не помнят продолжительности своего покоя. После реконструкции личности в конце эпохи те, кто проспал пять тысяч лет, будут реагировать не иначе, чем те, кто отдыхал пять дней. За исключением этой временн\'ой задержки, эти выжившие проходят режим восхождения так же, как и те, кто избежал более или менее продолжительного сна смерти.
\vs p030 4:13 Эти диспенсационные классы пилигримов миров используются для групповой моронтийной деятельности в работе локальных вселенных. Мобилизация таких огромных групп даёт большое преимущество; таким образом, они остаются вместе на долгие периоды эффективного служения.
\vs p030 4:14 \li{3.}\bibemph{Студенты обительских миров}. Все выжившие смертные, пробуждающиеся на обительских мирах, принадлежат к этому классу.
\vs p030 4:15 Физическое тело смертной плоти не является частью процесса реконструкции [reassembly] спящего выжившего; физическое тело возвратилось в прах. Назначенный серафим подготавливает [sponsors] новое тело, моронтийную форму, как новое жизненное средство\fnst{То есть новое средство самовыражения для той же самой личности.} для бессмертной души и для обитания вернувшегося Настройщика. Настройщик является хранителем духовной копии разума спящего выжившего. Назначенный серафим является хранителем сохранившейся индивидуальности~--- бессмертной души~--- в той мере, в какой она сформировалась. И когда эти двое~--- Настройщик и серафим~--- воссоединяют доверенные им собственности личности, новый индивидуум представляет собой воскрешение старой личности, выживание развивающейся моронтийной индивидуальности души. Такое воссоединение души и Настройщика вполне правильно называть воскресением, реконструкцией личностных факторов; но даже это не полностью объясняет повторное появление уцелевшей \bibemph{личности}. Хотя ты, вероятно, никогда не поймёшь факта такого необъяснимого процесса, ты когда\hyp{}нибудь эмпирически познаешь его истинность, если только не откажешься от плана выживания смертных.
\vs p030 4:16 \pc План первоначального содержания смертного под арестом на семи мирах постепенной подготовки действует почти универсально в Орвонтоне. В каждой локальной системе, состоящей примерно из тысячи обитаемых планет, есть семь обительских миров, обычно спутников или вторичных спутников\fnst{Спутников спутников.} столицы системы. Они являются принимающими мирами для большинства восходящих смертных.
\vs p030 4:17 Иногда все учебные миры, где обитают смертные, называют вселенскими <<обителями>>, и именно такие сферы имел в виду Иисус, когда сказал: <<В доме моего Отца много обителей>>. Отсюда и далее, в пределах одной группы сфер, таких как обительские миры, восходящие создания будут индивидуально продвигаться от одной сферы к другой и от одной фазы жизни к другой, но от одной ступени изучения вселенной к другой они всегда будут переходить классами.
\vs p030 4:18 \li{4.}\bibemph{Моронтийные прогрессоры}. Начиная с обительских миров и далее~--- через сферы системы, созвездия и вселенной,~--- смертные причисляются к классу моронтийных прогрессоров; они проходят переходные сферы восхождения смертных. Поднимаясь от низших моронтийных миров к высшим, восходящие смертные выполняют бесчисленные задания вместе со своими учителями и в компании своих дальше продвинувшихся и старших братьев.
\vs p030 4:19 Моронтийное развитие относится к непрерывному развитию интеллекта, духа и формы личности. Выжившие по\hyp{}прежнему остаются существами троякой природы. На протяжении всего моронтийного опыта они являются подопечными локальной вселенной. Режим сверхвселенной не действует до тех пор, пока не начнётся путь духа.
\vs p030 4:20 Смертные обретают настоящую индивидуальность духа непосредственно перед тем, как покинуть столицу локальной вселенной и отправиться в принимающие миры малых секторов сверхвселенной. Переход от заключительной моронтийной стадии к первому или низшему статусу духа представляет собой лишь небольшое изменение. Разум, личность и характер не меняются в результате такого продвижения; изменяется только форма. Но форма духа так же реальна, как и моронтийное тело, и не менее различима.
\vs p030 4:21 Прежде чем покинуть свои родные локальные вселенные и отправиться в принимающие миры сверхвселенной, смертные времени получают подтверждение духа от Сына Создателя и Материнского Духа локальной вселенной. С этого момента статус восходящего смертного утверждён навсегда. Никогда ещё подопечные сверхвселенной не сбивались с пути. Восходящие серафимы также продвигаются в ангельском статусе во время своего ухода из локальных вселенных.
\vs p030 4:22 \li{5.}\bibemph{Подопечные сверхвселенных}. Все восходящие существа, прибывающие на подготовительные миры сверхвселенных, становятся подопечными От Века Древних; они прошли моронтийную жизнь локальной вселенной и теперь стали признанными духами. Как молодые духи, они начинают восхождение в системе обучения и культуры сверхвселенной, начиная от принимающих сфер своего малого сектора через учебные миры десяти больших секторов и далее к высшим культурным сферам столицы сверхвселенной.
\vs p030 4:23 Существуют три категории духов\hyp{}учащихся, проживающих, соответственно, на столичных мирах развития духа малого сектора, больших секторов и сверхвселенной. Как восходящие моронтийные существа учились и работали на мирах локальной вселенной, так и восходящие духи продолжают осваивать новые миры, практикуясь в передаче другим того, что они впитали из эмпирических источников мудрости. Но посещение школы духовным существом в сверхвселенной очень непохоже ни на что из того, что когда\hyp{}либо появлялось в сферах воображения материального разума человека.
\vs p030 4:24 Прежде, чем покинуть сверхвселенную и отправиться в Хавону, эти восходящие духи проходят такой же основательный курс по управлению сверхвселенной, как и курс по управлению локальной вселенной, который они получили во время своего моронтийного опыта. Пока духи\hyp{}смертные не достигнут Хавоны, их главным предметом изучения, но не исключительным занятием, является овладение искусством управления локальной вселенной и сверхвселенной. Причина приобретения всего этого опыта пока не полностью ясна, но, вне всякого сомнения, прохождение такой учёбы есть дело благоразумное и необходимое в свете их возможного будущего предназначения в качестве членов Корпуса Завершения.
\vs p030 4:25 Режим сверхвселенной не является одинаковым для всех восходящих смертных. Они получают одинаковое общее образование, но специальные группы и классы проходят специальные курсы обучения и проводятся через особые курсы подготовки.
\vs p030 4:26 \li{6.}\bibemph{Пилигримы Хавоны}. Когда развитие духа завершено, даже если не до конца, выживший смертный готовится к долгому перелёту в Хавону~--- надёжную гавань\fnst{В английском игра слов: Havona \ldots\ haven.} эволюционных духов. На земле ты был созданием из плоти и крови; в пределах локальной вселенной ты был моронтийным существом; через сверхвселенную ты проходил эволюционирующим духом; с прибытием на принимающие миры Хавоны твоё духовное образование начинается всерьёз и реально; твоё конечное появление на Рае будет в виде совершенного духа.
\vs p030 4:27 Путешествие из столицы сверхвселенной в принимающие сферы Хавоны всегда совершается в одиночку. Отныне обучение в классе или группе больше не ведётся. Ты закончил техническую и административную подготовку эволюционных миров времени и пространства. Теперь начинается твоё \bibemph{личное образование,} твоё индивидуальное духовное обучение. От начала до конца, по всей Хавоне, обучение~--- личное и тройственное по своему характеру: интеллектуальное, духовное и эмпирическое.
\vs p030 4:28 Первым делом на твоём пути в Хавоне будет узнать и поблагодарить своего транспортного секонафима за долгое и безопасное путешествие. Затем тебя представят тем существам, которые поддержат твою раннюю деятельность в Хавоне. Затем ты отправишься зарегистрировать своё прибытие и подготовишь послание благодарности и поклонения для отправки Сыну Создателю твоей локальной вселенной~--- вселенскому Отцу, который сделал возможным твой путь сыновства. На этом завершаются формальности прибытия в Хавону; после чего тебе предоставляется длительный период досуга для свободного наблюдения, и это даёт возможность найти своих друзей, товарищей и партнёров по долгому опыту восхождения. По системе трансляции ты также сможешь узнать, кто из твоих собратьев\hyp{}пилигримов отправился в Хавону с того времени, как ты покинул Уверсу.
\vs p030 4:29 Факт твоего прибытия на принимающие миры Хавоны будет должным образом передан в столицу твоей локальной вселенной и сообщён лично твоему серафическому хранителю, где бы тот серафим ни находился.
\vs p030 4:30 Восходящие смертные были тщательно обучены делам эволюционных миров пространства; теперь они начинают свой долгий и плодотворный контакт с созданными сферами совершенства. Какую прекрасную подготовку к некоей будущей работе даёт этот комбинированный, уникальный и необыкновенный опыт! Но я не могу рассказать тебе о Хавоне; ты должен увидеть эти миры, чтобы оценить их великолепие и понять их величие.
\vs p030 4:31 \li{7.}\bibemph{Прибывшие в Рай}. Достигнув Рая со статусом постоянного обитателя, ты начинаешь изучать последовательный курс божественности и абсолютности. Твоё проживание на Рае означает, что ты нашёл Бога и должен быть зачислен в Смертный Корпус Завершения. Из всех созданий большой вселенной только те, кто слились с Отцом, принимаются в Смертный Корпус Завершения. Только такие индивидуумы принимают клятву завершителя. Другие существа Райского совершенства или достижений могут быть временно прикреплены к этому корпусу завершения, но они не принадлежат вечному назначению для неизвестной и нераскрытой миссии этого растущего воинства эволюционных и ставших совершенными ветеранов времени и пространства.
\vs p030 4:32 Прибывшим в Рай предоставляется период свободы, после которого они начинают своё общение с семью группами первичных супернафимов. Пройдя курс с руководителями поклонения они называются Райскими выпускниками, а затем, как завершители, назначаются на службу наблюдения и сотрудничества в разных концах обширного творения. Похоже, что пока для Смертного Корпуса Завершителей нет определённой или постоянной работы, хотя они служат во многих качествах в мирах, утверждённых в свете и жизни.
\vs p030 4:33 Даже если у Смертного Корпуса Завершения не будет будущего или нераскрытого предназначения, нынешнее назначение этих восходящих существ будет вполне адекватным и достойным. Их нынешнее предназначение полностью оправдывает всеобщий план эволюционного восхождения. Но будущие эпохи эволюции сфер внешнего пространства несомненно будут дальше развивать и с большей полнотой божественно освещать мудрость и исполненную любви доброту Богов в исполнении их божественного плана человеческого выживания и восхождения смертных.
\vs p030 4:34 \pc Это повествование вместе, с тем, что уже было раскрыто тебе ранее, и тем, что ты сможешь почерпнуть в связи с наставлением относительно твоего собственного мира, даёт общую картину пути восходящего смертного. Изложение значительно отличается в разных сверхвселенных, но этот пересказ даёт представление об обычном\fnst{Буквально <<среднем>>. Англ. average.} плане развития смертных, действующем в локальной вселенной Небадон и в седьмом сегменте большой вселенной~--- сверхвселенной Орвонтон.
\vsetoff
\vs p030 4:35 [При поддержке Могущественного Посланника из Уверсы.]
\quizlink
